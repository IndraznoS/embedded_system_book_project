\chapter{Introduction to Embedded Systems}

\section{What is an Embedded System?}

An embedded system is a computer system with a dedicated function within a larger mechanical or electrical system. Unlike general-purpose computers, embedded systems are designed to perform specific tasks, often with real-time computing constraints.

\subsection{Characteristics of Embedded Systems}

Embedded systems typically have the following characteristics:

\begin{itemize}
    \item \textbf{Dedicated Function:} Designed for specific tasks
    \item \textbf{Real-time Operation:} Must respond to inputs within specified time constraints
    \item \textbf{Resource Constraints:} Limited memory, processing power, and energy
    \item \textbf{Reliability:} Must operate continuously without failure
    \item \textbf{Cost Sensitivity:} Often produced in high volumes with tight cost constraints
\end{itemize}

\section{Applications of Embedded Systems}

Embedded systems are found in numerous applications across various industries:

\begin{itemize}
    \item \textbf{Consumer Electronics:} Smartphones, digital cameras, smart TVs
    \item \textbf{Automotive:} Engine control units, anti-lock braking systems, infotainment systems
    \item \textbf{Industrial Control:} PLCs, robotics, manufacturing automation
    \item \textbf{Medical Devices:} Pacemakers, insulin pumps, diagnostic equipment
    \item \textbf{IoT Devices:} Smart home devices, wearables, environmental sensors
\end{itemize}

\section{Arduino Platform}

Arduino is an open-source electronics platform based on easy-to-use hardware and software. It's ideal for learning embedded systems programming because:

\begin{itemize}
    \item Simple, accessible hardware platform
    \item Easy-to-understand programming language based on C/C++
    \item Large community and extensive documentation
    \item Wide range of sensors and actuators available
\end{itemize}

\subsection{Basic Arduino Program Structure}

An Arduino program (sketch) consists of two main functions:

\begin{lstlisting}[caption=Basic Arduino Structure]
void setup() {
    // Initialization code runs once
    pinMode(LED_BUILTIN, OUTPUT);
}

void loop() {
    // Main code runs repeatedly
    digitalWrite(LED_BUILTIN, HIGH);
    delay(1000);
    digitalWrite(LED_BUILTIN, LOW);
    delay(1000);
}
\end{lstlisting}

\section{Development Environment}

This book uses PlatformIO as the development environment for Arduino projects. PlatformIO offers several advantages over the traditional Arduino IDE:

\begin{itemize}
    \item Professional IDE integration (VS Code, CLion, etc.)
    \item Advanced debugging capabilities
    \item Library management
    \item Support for multiple platforms and frameworks
    \item Command-line interface for automation
\end{itemize}

\section{Summary}

In this chapter, we introduced the fundamentals of embedded systems, their characteristics, and applications. We also introduced the Arduino platform and PlatformIO development environment that will be used throughout this book.
