\documentclass[12pt,a4paper]{article}
\usepackage[utf8]{inputenc}
\usepackage[indonesian]{babel}
\usepackage{enumitem}
\usepackage{geometry}
\geometry{margin=2.5cm}
\usepackage{titlesec}
\usepackage{tabularx}
\usepackage{makecell}
\usepackage{array}
\usepackage{mathtools}
\usepackage{amsmath}
\newcolumntype{Y}{>{\raggedright\arraybackslash}X}

\titleformat{\section}{\large\bfseries}{\thesection}{1em}{}
\titleformat{\subsection}{\normalsize\bfseries}{\thesubsection}{1em}{}

\title{Rencana Kerja IoT Research Center \\ Tahun 2026--2027}
\author{Tim IoT Research Center \\ Politeknik Negeri Malang}
\date{Mei 2025}

\begin{document}

\maketitle

\section{Ringkasan Peraturan Direktur Politeknik Negeri Malang Nomor 3 Tahun 2024}

Berikut adalah ringkasan poin-poin penting dari Peraturan Direktur Politeknik Negeri Malang Nomor 3 Tahun 2024 tentang Pengelolaan \textit{Research Center} dan \textit{Research Group}, khususnya yang relevan dengan penyusunan rencana kerja dua tahunan:

\begin{enumerate}[leftmargin=*]
    \item \textbf{Kedudukan dan Tujuan}
          \begin{enumerate}
              \item \textit{Research Center} berkedudukan di bawah Pusat Penelitian dan Pengabdian kepada Masyarakat (P3M) dan dibentuk di tingkat institusi.
              \item \textit{Research Group} berkedudukan di bawah \textit{Research Center} dan dibentuk di tingkat Jurusan.
              \item Keduanya bertujuan untuk:
                    \begin{enumerate}
                        \item Melaksanakan penelitian unggulan (dasar, terapan, dan/atau pengembangan) dalam bidang ilmu pengetahuan, teknologi, kependidikan, sosial, dan budaya.
                        \item Menghasilkan produk inovasi unggulan berteknologi mutakhir yang berdampak nasional dan internasional.
                        \item Mendukung keberadaan \textit{teaching factory}.
                        \item Mengembangkan kemampuan dosen dalam penelitian serta menjadi sarana peningkatan kompetensi mahasiswa.
                    \end{enumerate}
          \end{enumerate}

    \item \textbf{Bidang Keahlian}
          \begin{enumerate}
              \item \textit{Research Center} harus memiliki bidang keahlian tertentu dengan pendekatan multidisiplin dan/atau interdisiplin.
              \item \textit{Research Group} memiliki bidang keahlian tertentu dengan pendekatan monodisiplin atau multidisiplin di tingkat jurusan.
              \item Bidang keahlian menjadi dasar pembeda antar \textit{Research Center} dan/atau \textit{Research Group}.
          \end{enumerate}

    \item \textbf{Program Kegiatan}
          \begin{enumerate}
              \item Kegiatan utama meliputi:
                    \begin{enumerate}
                        \item Perumusan peta jalan (\textit{roadmap}) pengembangan.
                        \item Penelitian, studi, dan/atau kajian ilmiah untuk menyelesaikan permasalahan masyarakat.
                        \item Pelatihan berbasis hasil penelitian.
                        \item Pelayanan jasa keilmuan sesuai bidang keahlian.
                        \item Koordinasi pendayagunaan peneliti.
                        \item Kerja sama dengan instansi pemerintah, swasta, dan lembaga dalam maupun luar negeri.
                        \item Partisipasi aktif dalam kompetisi hibah penelitian dan pengabdian di luar dana DIPA/swadana institusi.
                        \item Evaluasi kegiatan secara berkala.
                        \item Pengembangan kapasitas profesional peneliti dan staf pendukung.
                    \end{enumerate}
              \item Penelitian harus:
                    \begin{enumerate}
                        \item Mengacu pada peta jalan institusi dan memiliki target pencapaian yang jelas, terukur, dan realistis.
                        \item Menghasilkan luaran seperti: publikasi ilmiah terindeks/bereputasi, Kekayaan Intelektual, prototipe R\&D, prototipe industri, teknologi tepat guna, atau produk inovasi.
                        \item Dilakukan diseminasi hasil kepada internal maupun eksternal.
                        \item Diarahkan pada hilirisasi dan/atau komersialisasi.
                        \item Memberikan kontribusi pada pengelolaan \textit{teaching factory}.
                    \end{enumerate}
              \item Pelatihan dan pelayanan jasa keilmuan harus:
                    \begin{enumerate}
                        \item Berbasis hasil penelitian.
                        \item Mengacu pada kurikulum/silabus atau acuan teknis yang baku.
                        \item Dikoordinasikan dengan unit terkait (jurusan, prodi, atau bidang kerja sama).
                    \end{enumerate}
          \end{enumerate}

    \item \textbf{Target Kinerja Wajib}
          \begin{enumerate}
              \item Untuk \textit{Research Center} pada dua tahun pertama:
                    \begin{enumerate}
                        \item Minimal 2 luaran penelitian (jurnal internasional terindeks, paten, prototipe, atau produk inovasi).
                        \item Kerja sama lintas jurusan dan lintas institusi.
                        \item Kerja sama dengan pihak eksternal (pemerintah/swasta/lembaga).
                        \item Pengajuan hibah eksternal setiap tahun.
                        \item Implementasi hasil penelitian melalui PKM atau komersialisasi.
                    \end{enumerate}
              \item Untuk \textit{Research Center} setelah tahun kedua:
                    \begin{enumerate}
                        \item Minimal 4 luaran penelitian per periode dua tahun.
                        \item Harus memperoleh pendanaan hibah eksternal (bukan hanya mengajukan).
                        \item Implementasi wajib melalui PKM atau komersialisasi.
                    \end{enumerate}
              \item Evaluasi pencapaian target dilakukan setiap 2 tahun.
          \end{enumerate}

    \item \textbf{Pendanaan dan Fasilitas}
          \begin{enumerate}
              \item Pendanaan pengelolaan berasal dari anggaran P3M.
              \item Pendanaan kegiatan penelitian dapat berasal dari pemerintah, swasta, lembaga nirlaba, atau sponsor.
              \item Fasilitas yang disediakan institusi untuk \textit{Research Center} meliputi ruang sekretariat dan laboratorium/bengkel/studio sesuai kebutuhan.
          \end{enumerate}
\end{enumerate}

% --- Lanjutkan dengan bagian berikutnya: Visi, Misi, Roadmap, Program Kerja, dst. ---

\section{Target Kinerja IoT Research Center (2026--2027)}

Berdasarkan ketentuan dalam Peraturan Direktur Nomor 3 Tahun 2024, khususnya Pasal 17 ayat (2), IoT Research Center—yang telah beroperasi lebih dari dua tahun—wajib memenuhi target kinerja dua tahunan sebagai berikut:

\begin{enumerate}[leftmargin=*]
    \item \textbf{Luaran Penelitian}
          \begin{enumerate}
              \item Menghasilkan minimal \textbf{4 (empat) luaran penelitian} dalam periode 2026--2027, yang dapat berupa:
                    \begin{enumerate}
                        \item Publikasi pada jurnal internasional bereputasi/terindeks (Scopus Q1--Q4, Web of Science, atau DOAJ bereputasi),
                        \item Paten atau paten sederhana yang telah \textit{granted},
                        \item Prototipe R\&D atau prototipe industri berbasis IoT,
                        \item Teknologi Tepat Guna (TTG) berbasis IoT,
                        \item Produk inovasi siap komersial (misalnya: alat monitoring berbasis sensor, sistem otomasi, platform IoT terbuka).
                    \end{enumerate}
          \end{enumerate}

    \item \textbf{Kolaborasi dan Kerja Sama}
          \begin{enumerate}
              \item Mengembangkan kerja sama \textbf{lintas jurusan} (minimal 2 jurusan di luar asal anggota inti).
              \item Menjalin kerja sama \textbf{lintas institusi} (minimal 1 perguruan tinggi dalam negeri atau luar negeri).
              \item Menjalin kerja sama strategis dengan \textbf{pihak eksternal} (pemerintah daerah, BUMN, UMKM, atau perusahaan teknologi) sebanyak minimal 2 mitra.
          \end{enumerate}

    \item \textbf{Pendanaan Eksternal}
          \begin{enumerate}
              \item Memperoleh minimal \textbf{1 pendanaan hibah eksternal} (bukan hanya mengajukan) dari sumber di luar DIPA/swadana institusi, seperti:
                    \begin{enumerate}
                        \item Kemenristek/BRIN (misalnya: Penelitian Terapan Unggulan Perguruan Tinggi),
                        \item LPDP, Dikti, atau lembaga internasional (Erasmus+, AUN, dll.),
                        \item CSR perusahaan teknologi atau industri.
                    \end{enumerate}
          \end{enumerate}

    \item \textbf{Implementasi Hasil Penelitian}
          \begin{enumerate}
              \item Melaksanakan \textbf{hilirisasi} atau \textbf{komersialisasi} hasil penelitian melalui:
                    \begin{enumerate}
                        \item Program Pengabdian kepada Masyarakat (PKM) berbasis IoT,
                        \item Inkubasi bisnis atau \textit{spin-off} startup teknologi,
                        \item Adopsi oleh mitra industri atau pemerintah daerah.
                    \end{enumerate}
          \end{enumerate}

    \item \textbf{Pengembangan Kapasitas}
          \begin{enumerate}
              \item Menyelenggarakan minimal \textbf{2 kegiatan pelatihan} berbasis hasil riset IoT untuk dosen, mahasiswa, atau masyarakat.
              \item Melibatkan minimal \textbf{10 mahasiswa} dalam proyek riset (sebagai asisten peneliti atau tugas akhir terkait IoT).
          \end{enumerate}
\end{enumerate}

\section{Indikator Keberhasilan dan Cara Pengukuran}

Setiap target kinerja diukur melalui indikator objektif yang dapat diverifikasi. Berikut rinciannya:

\begin{enumerate}[leftmargin=*]
    \item \textbf{Luaran Penelitian}
          \begin{enumerate}
              \item \textit{Indikator}: Jumlah dan jenis luaran (jurnal, paten, prototipe, dll.).
              \item \textit{Cara Mengukur}:
                    \begin{enumerate}
                        \item Bukti publikasi (DOI, link jurnal, SINTA/Scopus/WoS screenshot),
                        \item Sertifikat paten dari DJKI,
                        \item Dokumentasi prototipe (foto, video, laporan teknis, uji coba),
                        \item Laporan komersialisasi atau MoU adopsi teknologi.
                    \end{enumerate}
          \end{enumerate}

    \item \textbf{Kolaborasi dan Kerja Sama}
          \begin{enumerate}
              \item \textit{Indikator}: Jumlah dan jenis mitra kolaborasi.
              \item \textit{Cara Mengukur}:
                    \begin{enumerate}
                        \item Surat pernyataan kerja sama (MoU/MoA),
                        \item Laporan kegiatan bersama (workshop, joint research, co-publication),
                        \item Daftar nama jurusan/institusi/mitra yang terlibat.
                    \end{enumerate}
          \end{enumerate}

    \item \textbf{Pendanaan Eksternal}
          \begin{enumerate}
              \item \textit{Indikator}: Jumlah dan nilai hibah yang diterima.
              \item \textit{Cara Mengukur}:
                    \begin{enumerate}
                        \item Surat keputusan penerimaan hibah,
                        \item Kontrak penelitian,
                        \item Bukti transfer dana atau laporan keuangan proyek.
                    \end{enumerate}
          \end{enumerate}

    \item \textbf{Implementasi Hasil Penelitian}
          \begin{enumerate}
              \item \textit{Indikator}: Tingkat penerapan hasil riset di luar lingkungan akademik.
              \item \textit{Cara Mengukur}:
                    \begin{enumerate}
                        \item Laporan PKM terverifikasi,
                        \item Bukti adopsi oleh mitra (surat penggunaan, testimoni, foto implementasi),
                        \item Dokumen inkubasi/startup (akta pendirian, pitch deck, pendanaan awal).
                    \end{enumerate}
          \end{enumerate}

    \item \textbf{Pengembangan Kapasitas}
          \begin{enumerate}
              \item \textit{Indikator}: Jumlah peserta pelatihan dan mahasiswa terlibat.
              \item \textit{Cara Mengukur}:
                    \begin{enumerate}
                        \item Daftar hadir pelatihan,
                        \item Sertifikat pelatihan,
                        \item Daftar mahasiswa asisten peneliti atau judul tugas akhir terkait IoT.
                    \end{enumerate}
          \end{enumerate}
\end{enumerate}

\section{Mekanisme Evaluasi}

Evaluasi pencapaian target kinerja dilakukan secara berkala dan transparan sesuai ketentuan Pasal 17 ayat (3) dan Pasal 6 ayat (1) huruf i:

\begin{enumerate}[leftmargin=*]
    \item \textbf{Jadwal Evaluasi}
          \begin{enumerate}
              \item Evaluasi internal: setiap \textbf{6 bulan} (Juni dan Desember) oleh tim internal IoT Research Center.
              \item Evaluasi eksternal: setiap \textbf{2 tahun} oleh P3M dan P2MPP berdasarkan dokumen capaian kinerja.
          \end{enumerate}

    \item \textbf{Tim Evaluasi}
          \begin{enumerate}
              \item Internal: Ketua dan anggota inti IoT Research Center.
              \item Eksternal: Tim dari P3M (fokus pada substansi dan luaran) dan P2MPP (fokus pada mutu dan kepatuhan terhadap SPMI).
          \end{enumerate}

    \item \textbf{Dokumen yang Dievaluasi}
          \begin{enumerate}
              \item Laporan kemajuan tahunan (termasuk bukti fisik dan digital),
              \item Dokumen kerja sama,
              \item Publikasi dan luaran riset,
              \item Laporan keuangan proyek (jika ada),
              \item Rekaman kegiatan (pelatihan, diseminasi, uji coba).
          \end{enumerate}

    \item \textbf{Output Evaluasi}
          \begin{enumerate}
              \item Rekomendasi perbaikan program,
              \item Penilaian kinerja (memenuhi/tidak memenuhi standar),
              \item Dasar pertimbangan kelanjutan atau peningkatan status Research Center.
          \end{enumerate}
\end{enumerate}


\section{Contoh Tabel Target Kinerja 2026--2027}

\small
\noindent
\begin{tabularx}{\textwidth}{|l|Y|Y|Y|}
    \hline
    \textbf{\makecell[l]{No.}}               &
    \textbf{\makecell[l]{Indikator Kinerja}} &
    \textbf{\makecell[l]{Baseline                                                                                                                               \\(2024--2025)}} &
    \textbf{\makecell[l]{Target                                                                                                                                 \\(2026--2027)}} \\
    % Kolom "Pencapaian" sengaja tidak dimasukkan di awal karena akan diisi nanti.
    % Jika ingin tampilkan sekarang sebagai placeholder:
    % & \textbf{\makecell[l]{Pencapaian\\(2026--2027)}} \\
    \hline
    1                                        & Jumlah luaran penelitian (jurnal terindeks, paten, prototipe IoT, produk inovasi) & 3 luaran   & 4 luaran        \\
    \hline
    2                                        & Jumlah publikasi internasional terindeks (Scopus/WoS/DOAJ bereputasi)             & 2 artikel  & 2 artikel       \\
    \hline
    3                                        & Jumlah paten/paten sederhana \textit{granted} oleh DJKI                           & 0          & 1               \\
    \hline
    4                                        & Jumlah prototipe IoT siap uji (R\&D atau industri)                                & 1          & 2               \\
    \hline
    5                                        & Jumlah kerja sama lintas jurusan                                                  & 2 jurusan  & 2+ jurusan      \\
    \hline
    6                                        & Jumlah kerja sama lintas institusi (PT dalam/luar negeri)                         & 1 mitra    & 1+ mitra        \\
    \hline
    7                                        & Jumlah kerja sama dengan pihak eksternal (pemerintah/swasta/UMKM)                 & 1 mitra    & 2 mitra         \\
    \hline
    8                                        & Pendanaan hibah eksternal yang diperoleh (bukan hanya diajukan)                   & 0          & $\geq$1 sumber  \\
    \hline
    9                                        & Implementasi hasil riset melalui PKM atau komersialisasi                          & 1 kegiatan & 2 kegiatan      \\
    \hline
    10                                       & Jumlah pelatihan berbasis riset IoT untuk dosen/mahasiswa/masyarakat              & 1 kegiatan & 2 kegiatan      \\
    \hline
    11                                       & Jumlah mahasiswa yang terlibat dalam proyek riset IoT                             & 6 orang    & $\geq$ 10 orang \\
    \hline
\end{tabularx}
\normalsize


\section{Fokus Tema Penelitian}

Berdasarkan roadmap strategis Pusat Riset IoT Politeknik Negeri Malang (2025–2045), tiga tema penelitian berikut dipilih sebagai fokus utama periode 2026–2027. Pemilihan tema ini selaras dengan visi \textit{“Mewujudkan Ekosistem Cerdas yang Berdampak”} dan memenuhi prinsip \textit{Start Small, Think Big} serta \textit{Technology for Justice}. Ketiga tema juga mendukung pencapaian target kinerja wajib sebagaimana diatur dalam Pasal 17 ayat (2) Peraturan Direktur Nomor 3 Tahun 2024, khususnya dalam hal:
\begin{enumerate}[leftmargin=*]
    \item Pengembangan infrastruktur riset berbasis kampus (\textit{campus-based testbed}),
    \item Kolaborasi lintas sektor (lingkungan, pemerintah desa/kota, akademik),
    \item Implementasi hasil riset melalui PKM atau komersialisasi,
    \item Penelitian terapan berbasis TKT 4–7 yang menghasilkan prototipe siap uji,
    \item Penguatan peran \textit{teaching factory} melalui integrasi riset dan pembelajaran.
\end{enumerate}

Berikut penjabaran ketiga tema beserta tujuan spesifiknya.

\subsection{Pengembangan Campus-Based IoT Infrastructure (C-IoT-TB)}

\textbf{Justifikasi:}
Tema ini merupakan fondasi teknis dan operasional seluruh kegiatan riset IoT di Polinema. Infrastruktur C-IoT-TB (Campus-based IoT Testbed) berfungsi sebagai \textit{living lab} yang mendukung eksperimen, pelatihan, dan pengembangan prototipe dalam lingkungan terkendali namun nyata. Pengembangan infrastruktur ini secara langsung menjawab amanat Pasal 3 ayat (1) huruf c dan Pasal 10 Peraturan Direktur No. 3/2024 tentang dukungan terhadap \textit{teaching factory} serta pemanfaatan laboratorium untuk aktivitas penelitian.

\textbf{Tujuan Spesifik:}
\begin{enumerate}[leftmargin=*]
    \item Memperluas jaringan sensor C-IoT-TB menjadi minimal 100 node aktif yang mencakup parameter lingkungan, energi, keamanan, dan utilitas kampus.
    \item Mengintegrasikan edge computing dan platform open-source (misalnya ThingsBoard, Node-RED) untuk pemrosesan data real-time.
    \item Menyediakan antarmuka API terbuka bagi mahasiswa dan dosen untuk pengembangan aplikasi riset.
    \item Menghubungkan C-IoT-TB dengan sistem manajemen energi dan keamanan kampus sebagai bagian dari \textit{smart campus}.
    \item Menjadi basis pelaksanaan minimal 5 proyek tugas akhir dan 2 kegiatan PKM berbasis IoT.
\end{enumerate}

\subsection{Participatory Environmental IoT Systems}

\textbf{Justifikasi:}
Tema ini menekankan pendekatan partisipatif dalam pemantauan lingkungan, di mana masyarakat (termasuk mahasiswa, staf, dan warga sekitar kampus) dilibatkan sebagai pengumpul, pengguna, dan pengambil keputusan berbasis data lingkungan. Hal ini selaras dengan prinsip \textit{Technology for Justice} dan mendukung prioritas nasional dalam pengelolaan lingkungan berkelanjutan. Tema ini juga memenuhi Pasal 6 ayat (1) huruf b dan Pasal 7 huruf g tentang penelitian untuk menyelesaikan masalah masyarakat dan implementasi melalui PKM.

\textbf{Tujuan Spesifik:}
\begin{enumerate}[leftmargin=*]
    \item Mengembangkan sistem pemantauan kualitas udara, kebisingan, dan mikroklimat berbasis sensor IoT yang terdistribusi di area kampus dan permukiman sekitar.
    \item Melibatkan minimal 200 peserta (mahasiswa, warga, sekolah) dalam pengumpulan dan interpretasi data lingkungan melalui platform digital partisipatif.
    \item Menghasilkan minimal 1 publikasi ilmiah internasional (Scopus Q2–Q3) tentang desain sistem IoT partisipatif untuk lingkungan perkotaan.
    \item Mengintegrasikan data lingkungan ke dalam kurikulum mata kuliah terkait (misalnya: Teknologi Lingkungan, Sistem Embedded).
    \item Melaksanakan 1 kegiatan PKM berbasis data lingkungan untuk mitigasi polusi di wilayah Malang Selatan.
\end{enumerate}

\subsection{Smart Village and City IoT Systems}

\textbf{Justifikasi:}
Tema ini menjembatani riset akademik dengan kebutuhan nyata pemerintah daerah, desa, dan kota dalam mewujudkan transformasi digital berbasis IoT. Fokus pada \textit{smart village} dan \textit{smart city} mendukung kebijakan nasional seperti \textit{Making Indonesia 4.0} dan Rencana Induk Riset Nasional (RIRN) bidang TIK. Tema ini juga memenuhi Pasal 6 ayat (1) huruf g dan Pasal 17 ayat (2) huruf c–e tentang kerja sama eksternal, pendanaan hibah, dan implementasi hasil riset.

\textbf{Tujuan Spesifik:}
\begin{enumerate}[leftmargin=*]
    \item Mengembangkan minimal 2 solusi IoT siap uji untuk desa/kota, seperti: sistem pemantau irigasi pintar, manajemen sampah berbasis sensor, atau dashboard tata kelola desa.
    \item Menjalin kerja sama formal dengan minimal 2 mitra (1 desa dan 1 OPD kota/kabupaten) untuk implementasi dan validasi lapangan.
    \item Mengajukan minimal 1 proposal hibah eksternal (misalnya: BRIN, Kemenkominfo, atau CSR perusahaan) yang berfokus pada IoT untuk pembangunan daerah.
    \item Melibatkan mahasiswa dalam program magang riset di lokasi mitra desa/kota.
    \item Mendokumentasikan model replikasi yang dapat diadopsi oleh perguruan tinggi vokasi lain di Indonesia.
\end{enumerate}


\section{Kolaborasi Multidisiplin dalam Mendukung Roadmap dan Fokus Penelitian}

Keberhasilan implementasi roadmap Pusat Riset IoT (2025–2045) dan pencapaian target kinerja periode 2026–2027 tidak dapat dicapai melalui pendekatan monodisiplin. Sebagaimana ditegaskan dalam Pasal 1 ayat (9) dan Pasal 5 ayat (1) Peraturan Direktur Politeknik Negeri Malang Nomor 3 Tahun 2024, \textit{Research Center} secara definisi dan fungsi harus mengembangkan keilmuan berdasarkan pendekatan \textbf{multidisiplin dan/atau interdisiplin}. Hal ini bukan sekadar formalitas administratif, melainkan kebutuhan substantif karena kompleksitas tantangan yang dihadapi—seperti transformasi digital kampus, ketahanan lingkungan, dan pembangunan desa pintar—tidak dapat diselesaikan hanya dengan keahlian teknik atau informatika semata.

\subsection{Justifikasi Kebutuhan Multidisiplin}

Pendekatan multidisiplin diperlukan karena:
\begin{enumerate}[leftmargin=*]
    \item \textbf{IoT adalah teknologi enabler, bukan solusi akhir.} Nilai IoT terwujud ketika diintegrasikan dengan domain aplikasi (lingkungan, tata kelola publik, kesehatan, pertanian), yang masing-masing memiliki logika, kebutuhan, dan etika tersendiri.
    \item \textbf{Peraturan Direktur mewajibkan kolaborasi lintas jurusan.} Pasal 17 ayat (2) huruf b secara eksplisit mensyaratkan adanya “kerjasama antar peneliti lintas jurusan dan lintas institusi” sebagai target kinerja wajib pasca-dua tahun.
    \item \textbf{Hilirisasi dan komersialisasi membutuhkan perspektif non-teknis.} Desain produk, keberlanjutan bisnis, penerimaan pengguna, dan kebijakan publik memerlukan kontribusi dari bidang manajemen, desain, hukum, dan ilmu sosial.
    \item \textbf{Prinsip “Technology for Justice” menuntut inklusivitas.} Solusi IoT yang adil harus mempertimbangkan aspek gender, aksesibilitas, kesenjangan digital, dan partisipasi masyarakat—ranah yang menjadi keahlian ilmu sosial-humaniora.
\end{enumerate}

\subsection{Contoh Kolaborasi Multidisiplin per Tema Penelitian}

\begin{enumerate}[leftmargin=*]
    \item \textbf{Pengembangan Campus-Based IoT Infrastructure (C-IoT-TB)}
          \begin{enumerate}
              \item \textit{Teknik Elektro}: Desain sensor node, manajemen daya, jaringan LoRaWAN/NB-IoT.
              \item \textit{Teknik Komputer}: Pengembangan firmware, protokol komunikasi, integrasi cloud.
              \item \textit{Sistem Informasi}: Pengembangan platform data, dashboard monitoring, analitik data.
              \item \textit{Teknik Sipil}: Penentuan lokasi penempatan sensor, analisis dampak struktural.
              \item \textit{Teknik Industri}: Optimasi operasional, manajemen proyek pengembangan infrastruktur.
              \item \textit{Ilmu Komunikasi}: Strategi diseminasi hasil riset dan edukasi pengguna kampus.
              \item \textit{Hukum}: Aspek privasi data, kepatuhan regulasi telekomunikasi.
              \item \textit{Ekonomi}: Analisis biaya-manfaat, model bisnis untuk pemeliharaan jangka panjang.
              \item   \textit{Psikologi}: Studi adopsi teknologi oleh pengguna kampus, faktor perilaku.
              \item  \textit{Desain Produk}: Desain antarmuka pengguna (UI/UX) untuk aplikasi monitoring.
              \item  \textit{Statistika}: Metodologi pengolahan dan analisis data sensor.
              \item \textit{Arsitektur/Tata Kota}: Integrasi sensor ke dalam tata ruang kampus dan infrastruktur fisik.
          \end{enumerate}

    \item \textbf{Participatory Environmental IoT Systems}
          \begin{enumerate}
              \item \textit{Teknik Informatika}: Pengembangan aplikasi mobile/web untuk partisipasi warga.
              \item \textit{Teknik Elektro}: Desain sensor kualitas udara, kebisingan, mikroklimat.
              \item \textit{Teknik Lingkungan}: Analisis parameter lingkungan, validasi data.
              \item \textit{Ilmu Lingkungan}: Interpretasi data lingkungan, dampak ekosistem.
              \item \textit{Ilmu Sosial}: Metodologi partisipasi masyarakat, analisis sosial.
              \item \textit{Statistika}: Analisis data partisipatif, deteksi anomali.
              \item \textit{Psikologi}: Studi motivasi dan perilaku partisipan.
              \item \textit{Desain Komunikasi Visual}: Desain antarmuka partisipatif, infografis.
              \item \textit{Ekonomi}: Analisis dampak ekonomi dari perbaikan kualitas lingkungan.
              \item \textit{Kesehatan Masyarakat}: Hubungan kualitas lingkungan dengan kesehatan warga.
              \item \textit{Geografi}: Pemetaan spasial data lingkungan.
              \item \textit{Sosiologi}: Studi dinamika komunitas dan perubahan sosial akibat intervensi teknologi.
          \end{enumerate}


    \item \textbf{Smart Village and City IoT Systems}
          \begin{enumerate}
              \item \textit{Teknik dan Teknologi}:
                    Teknik Komputer, Teknik Elektro, Teknik Sipil, Teknik Industri, Teknik Telekomunikasi, Teknik Mesin, Teknik Lingkungan, Teknik Perencanaan Wilayah dan Kota, Ilmu Komputer, Sistem Informasi — untuk desain infrastruktur fisik, jaringan, sensor, dashboard, dan sistem otomasi.

              \item \textit{Ilmu Sosial-Humaniora}:
                    Ilmu Pemerintahan, Sosiologi, Ilmu Komunikasi, Psikologi, Antropologi, Geografi — untuk memahami kebutuhan masyarakat, dinamika sosial, strategi partisipasi, dan adopsi teknologi di tingkat desa/kota.

              \item \textit{Ekonomi, Bisnis, dan Kebijakan}:
                    Ilmu Ekonomi, Manajemen, Administrasi Publik, Hukum — untuk menyusun model bisnis berkelanjutan, analisis kebijakan, tata kelola data, serta aspek regulasi dan privasi.

              \item \textit{Pertanian, Kesehatan, dan Lingkungan}:
                    Agribisnis, Teknologi Pangan, Kesehatan Masyarakat, Keperawatan, Biologi, Teknik Kimia — untuk aplikasi IoT dalam ketahanan pangan, pemantauan kesehatan, pengelolaan limbah, dan konservasi sumber daya alam.

              \item \textit{Bidang Pendukung Lainnya}:
                    Statistika, Ilmu Data, Pariwisata, Energi Terbarukan — untuk analitik data, pengembangan ekowisata digital, dan integrasi solusi energi berkelanjutan.
          \end{enumerate}
\end{enumerate}


\subsection{Mekanisme Operasional Kolaborasi Multidisiplin}

Untuk mewujudkan kolaborasi ini, Pusat Riset IoT akan:
\begin{enumerate}[leftmargin=*]
    \item Mengundang dosen dari jurusan non-teknik sebagai \textit{affiliated researchers} dalam proyek riset.
    \item Menyelenggarakan \textit{workshop tematik lintas jurusan} setiap semester untuk menyelaraskan bahasa teknis dan kebutuhan domain.
    \item Mengalokasikan dana internal untuk proposal riset kolaboratif multidisiplin.
    \item Menyusun \textit{joint supervision} tugas akhir mahasiswa yang melibatkan pembimbing dari minimal dua jurusan.
    \item Menggunakan C-IoT-TB sebagai \textit{common platform} yang dapat diakses oleh semua disiplin ilmu di Polinema.
\end{enumerate}

Dengan demikian, pendekatan multidisiplin bukan hanya memenuhi amanat regulasi, tetapi menjadi fondasi strategis untuk menciptakan inovasi IoT yang \textbf{teknis, berdampak, dan berkeadilan}.

\section{Tim Perumus}

Dokumen Roadmap Pusat Riset IoT Polinema ini ditetapkan pada:

\begin{center}
    \textbf{Politeknik Negeri Malang} \\
    \textit{Malang, 12 September 2025} \\
\end{center}

\noindent
Dengan ini, tim perumus menyatakan bahwa dokumen ini merupakan hasil konsensus dan kontribusi intelektual seluruh anggota, serta menyetujui isi dan arah strategisnya.

\vspace{2.2cm} % Spasi sebelum tabel

\begin{tabular}{|p{5cm}|p{8cm}|}
    \hline
    \textbf{Nama}           & \textbf{Tanda Tangan} \\
    \hline
    Indrazno Siradjuddin    &                       \\[1.2cm] % Ruang tanda tangan lebih tinggi
    \hline
    Erfan Rohadi            &                       \\[1.2cm]
    \hline
    Devi Mega Risdiana      &                       \\[1.2cm]
    \hline
    Rakhmat Arianto         &                       \\[1.2cm]
    \hline
    Vipkas Al Hadid Firdaus &                       \\[1.2cm]
    \hline
    Rudy Ariyanto           &                       \\[1.2cm]
    \hline
    Ahmadi Yuli Ananta      &                       \\[1.2cm]
    \hline
    Ade Ismail              &                       \\[1.2cm]
    \hline
    Usman Nurhasan          &                       \\[1.2cm]
    \hline
\end{tabular}

\vspace{2cm} % Spasi setelah tabel

\end{document}