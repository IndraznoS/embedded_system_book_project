\usepackage[a4paper, top=30mm, bottom=30mm, left=30mm, right=25mm]{geometry}
%Penggunaan bahasa
\usepackage[bahasa]{babel}
%Library penggunaan gambar
\usepackage{graphicx}
%\usepackage{subcaption}
\usepackage{siunitx}
%Pengaturan Judul
\usepackage[utf8]{inputenc}
%\usepackage[utf8x]{inputenc}
%\usepackage{import}
% Menambahkan dummy text, bullet, atau rumus
\usepackage{blindtext}
\usepackage{lipsum}
% Pengaturan bab
\usepackage{sectsty}
\usepackage{titlesec}
% Pengelolaan Daftar Pustaka
%\usepackage{natbib}
%\usepackage[nottoc]{tocbibind}
% Pengaturan Sub judul gambar atau tabel
\usepackage{subcaption}
% Pengaturan tabel
\usepackage{multirow}
\usepackage[table,xcdraw]{xcolor}
%Pengaturan Enumerate
\usepackage{enumitem}
%pengaturan judul gambar
\usepackage[labelsep=quad]{caption}
%pengaturan tabel berlanjut di halaman berikutnya
\usepackage{longtable}
\usepackage{amssymb}
\usepackage{amsmath}
\usepackage{appendix}
\usepackage{soul}
%pengaturan rata kiri kanan
\usepackage{ragged2e}
% Pengaturan spasi section
\titlespacing{\section}{0pt}{\parskip}{\parskip}

% Pengaturan left caption
%\captionsetup[table]{singlelinecheck=off}
\usepackage[font=normalsize,labelfont=bf,tableposition=top]{caption}
\usepackage{physics}
\usepackage[utf8]{inputenc}

\usepackage{booktabs} % To thicken table lines
\usepackage{float}

\usepackage{booktabs}
\usepackage{multirow}
\usepackage[table,xcdraw]{xcolor}

%Pengaturan kotak kosong
\usepackage{tcolorbox}
% Pengaturan Judul di Tengah
\titleformat{\chapter}[display]{\LARGE\bfseries\centering}{\MakeUppercase\chaptertitlename\ \thechapter}{-5pt}{\Large\bfseries\uppercase}
\titlespacing*{\chapter}{0pt}{-20pt}{20pt}
% Pengaturan ukuran huruf section
\titleformat*{\section}{\Large\bfseries}
\titleformat*{\subsection}{\large\bfseries}
\titleformat*{\subsubsection}{\large\bfseries}
\titleformat*{\paragraph}{\large\bfseries}
\titleformat*{\subparagraph}{\large\bfseries}

%Blankpage%
\def\blankpage{%
    \clearpage%
    \thispagestyle{empty}%
    \addtocounter{page}{-1}%
    \null%
    \clearpage}
%Pengaturan Roman dan Arabic untuk Chapter, Section, Subsection
\renewcommand{\thechapter}{\arabic{chapter}}
\renewcommand{\thesection}{\arabic{chapter}.\arabic{section}}
\renewcommand{\thesubsection}{\thesection.\arabic{subsection}}
\renewcommand{\thefigure}{\arabic{chapter}.\arabic{figure}}
\renewcommand{\thetable}{\arabic{chapter}.\arabic{table}}


%% Mengganti nama Bab menjadi Percobaan
\addto\captionsbahasa{\renewcommand{\chaptername}{BAB}}
\addto\captionsbahasa{\renewcommand{\bibname}{DAFTAR PUSTAKA}}

% Pengaturan Nama BAB di TOC
\usepackage[titles]{tocloft}
\setlength{\cftchapnumwidth}{0pt}
\setlength{\cftbeforechapskip}{\baselineskip}
\renewcommand{\cftchappresnum}{\chaptername\ }
\renewcommand{\cftchapaftersnum}{.}
\renewcommand{\cftchapaftersnumb}{\newline}
\renewcommand{\cftchapdotsep}{\cftdotsep}

\graphicspath{{././figures/}}

\tolerance=1
\emergencystretch=\maxdimen
\hyphenpenalty=10000
\hbadness=10000

\setlength{\parskip}{0.5em}


%\usepackage{xcolor}
\definecolor{tssteelblue}{RGB}{70,130,180}
\definecolor{tsorange}{RGB}{255,138,88}
\definecolor{tsblue}{RGB}{23,74,117}
\definecolor{tsforestgreen}{RGB}{21,122,81}
\definecolor{tsyellow}{RGB}{255,185,88}
\definecolor{tsgrey}{RGB}{200,200,200}
\definecolor{codegreen}{rgb}{0,0.6,0}
\definecolor{codegray}{rgb}{0.5,0.5,0.5}
\definecolor{codepurple}{rgb}{0.58,0,0.82}
\definecolor{backcolour}{rgb}{0.95,0.95,0.92}
\definecolor{lightgray}{rgb}{0.95,0.95,0.95}
\definecolor{lightyellow}{rgb}{0.98, 0.98, 0.7}

\usepackage{tcolorbox}
\usepackage{algorithm}
\usepackage{algpseudocode}
\usepackage{listings}
\tcbuselibrary{listings,breakable}
\usepackage{amsthm}

\newenvironment{boxprogram}
{\begin{tcolorbox}
        [enhanced jigsaw,breakable,pad at break*=1mm,
            colback=backcolour,boxrule=0pt,frame hidden,
            borderline west={1.5mm}{-2mm}{tsblue}]}
        {\end{tcolorbox}}

\newtheoremstyle{styleprogram}
{0pt}{0pt}{\normalfont}{0pt}{\small\bf\sffamily\color{black}}{\\}{0.25em}
{\small\sffamily\color{tsblue}\thmname{#1}
    \nobreakspace\thmnumber{#2}
    \thmnote{\nobreakspace\the\thm@notefont\sffamily\bfseries\color{black} (#3)}}

%\newtheoremstyle{styleprogram}
%{0pt}{0pt}{\normalfont}{0pt}{\small\bf\sffamily\color{black}}{\\}{0.25em}
%{\small\sffamily\color{tsblue}\thmname{#1}
% \nobreakspace\thmnumber{\@ifnotempty{#1}{}\@upn{#2}}
% \thmnote{\nobreakspace\the\thm@notefont\sffamily\bfseries\color{black} (#3)}}

\theoremstyle{styleprogram}
\newtheorem{envprogram}{Program}[chapter]

\newenvironment{program}
{\begin{boxprogram}\begin{envprogram}}
            {\end{envprogram}\end{boxprogram}}


\lstdefinestyle{mystyle}{
    backgroundcolor=\color{backcolour},
    commentstyle=\color{codegreen},
    keywordstyle=\color{magenta},
    numberstyle=\tiny\color{codegray},
    stringstyle=\color{codepurple},
    basicstyle=\ttfamily\scriptsize,
    breakatwhitespace=false,
    breaklines=false,
    captionpos=b,
    keepspaces=true,
    numbers=left,
    numbersep=5pt,
    showspaces=false,
    showstringspaces=false,
    showtabs=false,
    tabsize=2
}

\lstset{style=mystyle}

\usepackage{makecell}
\usepackage{hyperref}
\AtBeginDocument{\RenewCommandCopy\qty\SI}
