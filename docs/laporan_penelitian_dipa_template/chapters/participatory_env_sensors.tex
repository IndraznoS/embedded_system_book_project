\documentclass[12pt,a4paper]{article}
\usepackage[utf8]{inputenc}
\usepackage[T1]{fontenc}
\usepackage{lmodern}
\usepackage{geometry}
\usepackage{array}
\usepackage{booktabs}
\usepackage{longtable}
\usepackage{caption}
\usepackage{enumitem}
\usepackage{parskip}
\usepackage{ragged2e}

\geometry{margin=1in}
\setlength{\parindent}{0pt}

% Custom enumerate settings for consistent formatting
\setlist[enumerate,1]{label=\arabic*., left=0pt}
\setlist[enumerate,2]{label=\alph*., left=15pt}
\setlist[itemize,1]{left=0pt}
\setlist[itemize,2]{left=15pt}

\title{Environmental Sensors for Mobile End-Nodes in Participatory Sensing}
\author{}
\date{}

\begin{document}

\maketitle

For \textbf{mobile end-nodes} (e.g., smartphones, wearable devices, or portable IoT nodes carried by people) in a \textbf{participatory sensing scheme}, the choice of environmental sensors should balance \textit{accuracy}, \textit{power consumption}, \textit{size}, \textit{cost}, and \textit{relevance to urban or personal exposure monitoring}. Below are key environmental sensors commonly used or suitable for such applications.

\section*{Common Environmental Sensors for Mobile Participatory Sensing}

\begin{longtable}{>{\RaggedRight}p{3.5cm}>{\RaggedRight}p{2.8cm}>{\RaggedRight}p{4.5cm}>{\RaggedRight}p{4.5cm}}
    \toprule
    \textbf{Sensor Type}                                & \textbf{Measures}                                           & \textbf{Typical Use Cases}                                 & \textbf{Notes}                                                                       \\
    \midrule
    \endhead

    \bottomrule
    \endfoot

    \multicolumn{4}{l}{\textbf{Air Quality Sensors}}                                                                                                                                                                                                                      \\
    \addlinespace

    PM2.5 / PM10 sensor                                 & Particulate matter (dust, smoke, pollen)                    & Urban pollution mapping, health studies                    & Low-cost optical sensors (e.g., SDS011, PMS5003) are popular but require calibration \\

    CO\textsubscript{2} sensor                          & Carbon dioxide concentration                                & Indoor air quality, ventilation assessment                 & NDIR-based sensors (e.g., SCD30, MH-Z19)                                             \\

    VOC / eCO\textsubscript{2} sensor                   & Volatile Organic Compounds / equivalent CO\textsubscript{2} & Indoor air pollution, chemical exposure                    & Metal-oxide (MOX) sensors (e.g., CCS811, SGP30); drift over time                     \\

    NO\textsubscript{2}, O\textsubscript{3}, CO sensors & Nitrogen dioxide, ozone, carbon monoxide                    & Traffic-related pollution, industrial exposure             & Electrochemical sensors; more expensive, need frequent calibration                   \\

    \addlinespace
    \multicolumn{4}{l}{\textbf{Meteorological Sensors}}                                                                                                                                                                                                                   \\
    \addlinespace

    Temperature \& Humidity                             & Ambient temperature and relative humidity                   & Heat stress, comfort index, data correction                & Integrated in most platforms (e.g., DHT22, SHT31)                                    \\

    Barometric pressure                                 & Atmospheric pressure                                        & Altitude estimation, weather trends                        & (e.g., BMP280, BME280)                                                               \\

    \addlinespace
    Noise sensor                                        & Sound pressure level (dB)                                   & Urban noise pollution, traffic/industrial noise            & Requires calibrated microphone + A-weighting filter (e.g., INMP441 + DSP)            \\

    Light / UV sensor                                   & Ambient light intensity, UV index                           & Sun exposure, circadian rhythm studies                     & (e.g., BH1750 for lux, VEML6075 for UV)                                              \\

    GPS module                                          & Location, speed, altitude                                   & Geotagging sensor data, mobility patterns                  & Essential for spatial mapping; high power use                                        \\

    Inertial sensors (often built-in)                   & Acceleration, orientation                                   & Activity recognition (walking, cycling), context awareness & Helps infer user context (e.g., indoors vs. outdoors)                                \\
\end{longtable}

\section*{Practical Considerations for Mobile End-Nodes}

\begin{enumerate}
    \item \textbf{Power Efficiency}
          \begin{enumerate}
              \item Choose low-power sensors and microcontrollers (e.g., ESP32, Arduino Nano 33 IoT).
              \item Implement sleep modes and efficient data logging strategies.
              \item Optimize sampling frequency based on study needs (e.g., every 1-5 minutes).
              \item Use local storage (e.g., microSD) to buffer data if real-time transmission is not feasible.
              \item Consider energy harvesting (e.g., solar panels) for extended deployments.
              \item Use power banks or rechargeable batteries for longer field use.
              \item Monitor battery levels and alert users when recharging is needed.
              \item Design for easy battery replacement or recharging.
              \item Minimize power-hungry components (e.g., GPS) by activating them only when necessary.
              \item Use efficient communication protocols (e.g., LoRa, BLE) to reduce transmission power.
              \item Balance data resolution with power consumption to extend operational time.
              \item Test and profile power consumption in real-world scenarios to optimize settings.
          \end{enumerate}


    \item \textbf{Size \& Integration}
          \begin{enumerate}
              \item Select compact, lightweight sensors suitable for wearables or handheld devices.
              \item Ensure proper sensor placement for accurate readings (e.g., avoid body heat interference for temperature sensors).
              \item Use modular designs to allow easy swapping of sensors based on study requirements.
              \item Consider waterproof or rugged enclosures for outdoor use.
              \item Ensure good airflow for air quality sensors to avoid measurement bias.
              \item Minimize the overall weight to enhance user comfort during prolonged use.
              \item Use flexible PCBs or wearable-friendly designs for better ergonomics.
              \item Ensure that the device is unobtrusive to encourage user compliance.
              \item Design for easy attachment to clothing or accessories (e.g., clips, lanyards).
              \item Consider the aesthetics of the device to increase user acceptance.
              \item Ensure that the device does not interfere with daily activities.
              \item Test the device in real-world conditions to ensure durability and reliability.
          \end{enumerate}

    \item \textbf{Calibration \& Data Quality}
          \begin{enumerate}
              \item low-cost sensors often require field calibration against reference-grade instruments.
              \item Implement quality control procedures to identify and filter out erroneous data.
              \item Use statistical methods or machine learning to correct sensor drift and improve accuracy.
              \item Regularly recalibrate sensors during long-term deployments.
              \item Validate sensor data with ground truth measurements when possible.
              \item Document calibration procedures and maintain logs for transparency.
              \item Use redundant sensors to cross-validate measurements.
              \item Monitor sensor performance over time to detect degradation.
              \item Train participants on proper device handling to minimize user-induced errors.
              \item Use data visualization tools to identify anomalies or trends in the data.
              \item Share calibration data and methods with the research community for reproducibility.
              \item Consider environmental factors (e.g., temperature, humidity) that may affect sensor readings
          \end{enumerate}

    \item \textbf{Privacy \& Ethics}
          \begin{enumerate}
              \item anonnymize personal data (e.g., GPS coordinates) to protect participant privacy.
              \item Ensure compliance with local regulations regarding data collection and storage.
              \item Obtain informed consent from participants, clearly explaining the purpose of data collection and how data will be used.
              \item Implement secure data transmission and storage protocols (e.g., encryption).
              \item Allow participants to withdraw from the study and delete their data if desired.
              \item Be transparent about data sharing with third parties or public databases.
              \item Minimize the collection of personally identifiable information (PII).
              \item Use aggregated data for analysis to further protect individual identities.
              \item Regularly review ethical considerations as the project evolves.
              \item Engage with community stakeholders to address concerns and ensure cultural sensitivity.
              \item Provide participants with access to their own data and study results.
              \item Establish a data governance framework to oversee ethical data use.
          \end{enumerate}

    \item \textbf{Communication}
          \begin{enumerate}
              \item Use Wi-Fi, Bluetooth Low Energy (BLE), or cellular (NB-IoT/LTE-M) to upload data to a cloud platform.
              \item Implement data compression techniques to reduce transmission size.
              \item Use local storage (e.g., microSD) to buffer data if real-time transmission is not feasible.
              \item Schedule data uploads during low-usage periods to minimize network congestion.
              \item Ensure robust error handling and retry mechanisms for data transmission.
              \item Use secure communication protocols (e.g., HTTPS, MQTT with TLS) to protect data in transit.
              \item Optimize data formats (e.g., JSON, CSV) for efficient parsing and storage.
              \item Consider using edge computing to preprocess data before transmission.
              \item Monitor network connectivity and provide feedback to users if uploads fail.
              \item Use adaptive transmission strategies based on network availability and power constraints.
              \item Test communication reliability in various environments (urban, rural, indoor).
              \item Provide clear instructions to participants on how to connect and sync their devices.
          \end{enumerate}
\end{enumerate}

\section*{Example Participatory Applications}

\begin{enumerate}
    \item \textbf{Urban Air Quality Mapping}: Citizens carry portable PM2.5 + GPS nodes while commuting.
    \item \textbf{Heat Vulnerability Studies}: Wearables with temperature/humidity sensors in elderly populations.
    \item \textbf{Noise Pollution Campaigns}: Smartphone-based noise logging during daily activities.
    \item \textbf{Personal Exposure Assessment}: Workers or cyclists monitor real-time CO/NO\textsubscript{2} along routes.
\end{enumerate}



\section*{Popular Mobile Sensing Platforms}

\begin{enumerate}
    \item \textbf{Smartphones}

          \begin{enumerate}
              \item Built-in sensors: GPS, accelerometer, gyroscope, magnetometer, microphone, light sensor.
              \item External sensors via Bluetooth (e.g., AirBeam, Atmotube).
              \item Apps for data logging and transmission (e.g., OpenSense, AirCasting).
              \item High user adoption but limited by built-in sensor quality and battery life.
              \item external sensors can enhance capabilities but may increase complexity and cost.
              \item Smartphones offer a familiar interface, making them accessible for a wide range of users.
              \item They can leverage existing connectivity options (Wi-Fi, cellular) for real-time data upload.
              \item Data privacy concerns must be addressed, especially with location tracking.
              \item Battery consumption can be high when using multiple sensors and GPS simultaneously.
              \item Regular software updates can improve functionality and security.
              \item Integration with cloud services allows for large-scale data aggregation and analysis.
              \item Custom apps can be developed to tailor data collection to specific research needs.
          \end{enumerate}

    \item \textbf{Custom Wearables}: Example: ESP32 + BME280 + PMS5003 + GPS in a small enclosure.

    \item \textbf{Open-source Kits}
          \begin  {enumerate}
    \item Air Quality Egg
    \item Atmotube
    \item PurpleAir (mobile version)
\end{enumerate}

\item \textbf{DIY Solutions}: Using Arduino, ESP32, or Raspberry Pi with modular sensors.
\begin{enumerate}
    \item Cost-effective and flexible for specific research needs.
    \item Requires technical skills for assembly, programming, and calibration.
    \item Can be tailored to include only necessary sensors, reducing size and power consumption.
    \item Open-source communities provide support and resources.
    \item May lack the polish and user-friendliness of commercial products.
    \item
    \item Highly customizable but requires technical expertise.
    \item Can be optimized for specific research needs and budgets.
    \item Community support and open-source resources available.
    \item May face challenges in terms of durability and user-friendliness.
    \item Allows for experimentation with different sensor combinations and configurations.
    \item Can be integrated with various communication modules (Wi-Fi, LoRa, GSM).
    \item Requires careful design to ensure power efficiency and data reliability.
    \item Testing and validation are crucial to ensure data quality.
    \item Can be a cost-effective solution for large-scale deployments.
    \item Encourages innovation and collaboration within the research community.
    \item Provides hands-on experience with sensor technology and data collection methods.
\end{enumerate}

\end{enumerate}

\section*{Recommended Minimum Viable Mobile Node}

For a cost-effective, battery-powered mobile end-node in participatory sensing:

\begin{enumerate}
    \item \textbf{Microcontroller}: ESP32 (Wi-Fi + BLE)

    \item \textbf{Sensors}
    \item BME280 (temperature, humidity, pressure)
    \item PMS5003 (PM2.5/PM10)
    \item SGP30 (VOC/eCO\textsubscript{2})
    \item INMP441 (digital MEMS microphone for noise)

    \item \textbf{GPS}: NEO-6M or PA6H

    \item \textbf{Power}: 18650 Li-ion battery + TP4056 charger

    \item \textbf{Enclosure}: Lightweight, breathable (to ensure proper airflow for air quality sensors)
\end{enumerate}

This setup enables rich environmental monitoring while remaining portable and affordable (approximately \$50–80 per unit).

\end{document}