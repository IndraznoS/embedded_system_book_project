\chapter{Pendahuluan}
\section{Latar Belakang}

\section{Rumusan Masalah}
Perumusan masalah dalam penelitian adalah langkah penting yang mengarahkan fokus studi secara spesifik. Untuk penelitian dengan judul "Disain Dan Implementasi Perangkat Pembelajaran Sistem Otomasi Industri Dengan Model Predictive Control: Studi Kasus Cart-Inverted Pendulum", permasalahan dapat dirumuskan sebagai berikut:


\section{Batasan Masalah}
Untuk memfokuskan penelitian pada pengembangan perangkat keras dengan menggunakan sistem tertanam seperti Arduino atau ESP32 dan software untuk program sistem tertanam tersebut, batasan masalah harus dijelaskan secara eksplisit. Berikut adalah beberapa batasan masalah yang bisa ditetapkan:



\section{Tujuan Penelitian}
Tujuan dari penelitian ini adalah untuk merancang dan mengimplementasikan sebuah perangkat pembelajaran yang berfokus pada aplikasi Model Predictive Control (MPC) dalam sistem dinamik khususnya Cart-Inverted Pendulum menggunakan platform mikrokontroler seperti Arduino atau ESP32. Upaya ini bertujuan untuk memfasilitasi pemahaman konseptual yang lebih baik dan penerapan praktis MPC bagi mahasiswa dan peneliti, dengan menyediakan alat praktikum yang mampu mengilustrasikan tantangan nyata sistem kontrol non-linear dan tak stabil. Perangkat pembelajaran ini diharapkan dapat menjadi alat efektif dalam mengatasi kesenjangan antara teori kontrol yang kompleks dengan aplikasi dunia nyata, sekaligus meningkatkan kualitas pendidikan teknik kontrol dan mendukung kurikulum yang ada dengan cara yang inovatif dan interaktif.

\begin{table}[!htb]
    %\centering
    \caption{Luaran Penelitian}
    \label{tab:my-table}
    \resizebox{\textwidth}{!}{%
        \begin{tabular}{|c|l|c|}
            \hline
            \textbf{No} & \multicolumn{1}{c|}{\textbf{Jenis Luaran}}                                    & \textbf{Indikator Capaian *)} \\ \hline
            \multicolumn{3}{|l|}{Luaran Wajib}                                                                                          \\ \hline
            1           & \begin{tabular}[c]{@{}l@{}}IPTEKSB berupa prototype\end{tabular}              & Ada                           \\ \hline
            2           & Feasibility study (analisis kebutuhan pelanggan, kondisi pasar, dan teknis)   & Ada                           \\ \hline
            3           & Dokumen uji yang dikeluarkan/ telah divalidasi Ada oleh lembaga yang kompeten & Ada                           \\ \hline
            4           & Dokumen Implementation Arrangement (IA)                                       & Ada                           \\ \hline
            \multicolumn{3}{|l|}{Luaran Tambahan}                                                                                       \\ \hline
            1           & Prosiding seminar internasional terindeks (SCOPUS, IEEE)                      & Presented                     \\ \hline
        \end{tabular}%
    }
\end{table}

\section{Manfaat Penelitian}


\section{Luaran Penelitian}
Luaran penelitian dapat dilihat pada Tabel~\ref{tab:my-table}.






