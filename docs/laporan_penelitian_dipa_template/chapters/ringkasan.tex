\addcontentsline{toc}{chapter}{Ringkasan}
\title{Ringkasan}
\begin{center}
	\large \textbf{RINGKASAN}
\end{center}

Penelitian dengan judul “Disain Dan Implementasi Perangkat Pembelajaran Sistem Otomasi Industri Dengan Model Predictive Control: Studi Kasus Cart-Inverted Pendulum” berupaya mengembangkan sebuah prototipe sistem kontrol berbasis Model Predictive Control (MPC) yang dapat diaplikasikan dalam sebuah laboratorium sistem kontrol. Pada tahap awal penelitian, asumsi baseline adalah bahwa teknologi yang ada telah mencapai TKT level 4, yang menggambarkan adanya literatur dan pemahaman konseptual yang cukup kuat terkait MPC. Namun, perlu adanya pengembangan lebih lanjut agar teknologi ini dapat ditingkatkan hingga mencapai TKT level 6 di akhir penelitian. Mencapai TKT level 6 dalam kesiapan teknologi akan berarti adanya implementasi prototipe MPC yang berfungsi penuh dan teruji dengan baik. Hal ini akan melibatkan pengembangan perangkat keras yang solid serta pengoptimalan algoritma pendukungnya. Secara terperinci, proses penelitian akan dimulai dengan pemilihan komponen perangkat keras yang tepat untuk implementasi MPC. Pengembangan perangkat lunak untuk mendukung algoritma kontrol juga akan menjadi fokus yang krusial. Keselarasan antara perangkat keras dan perangkat lunak, serta optimalisasi kinerja sistem secara keseluruhan akan menjadi bagian penting dalam pembangunan prototipe ini.  Seperti yang telah disebutkan sebelumnya, output yang diharapkan dari penelitian ini sangat komprehensif. Pertama-tama, prototipe cart-inverted pendulum dengan kontrol MPC yang berfungsi penuh akan menjadi pencapaian kunci, menandai sebuah langkah maju dalam penggabungan teori kontrol yang rumit dengan implementasi dunia nyata. Selain itu, artikel ilmiah yang dikembangkan dalam penelitian ini diharapkan akan diterima untuk presentasi pada konferensi internasional terindeks seperti IEEE (ICVE2024). Hal ini akan menjadi suatu pencapaian signifikan yang dapat memberi pengakuan dan kontribusi pengetahuan yang substansial dalam bidang sistem kontrol. Selain itu, penelitian memasukkan juga pengembangan dokumen Feasibility Study serta dokumen Implementation Arrangement dan Hak Cipta berupa KI program komputer. Ini menunjukkan keseriusan dalam menjelajahi potensi komersialisasi prototipe hasil penelitian ini. Estimasi mengenai perhitungan biaya, analisis pasar, serta proyeksi pendapatan akan menjadi bagian utama dalam dokumen feasibility study. Dokumen arrangement implementasi akan memberikan panduan terinci tentang bagaimana prototipe ini dapat diterapkan secara komersial serta menjelaskan langkah demi langkah yang diperlukan untuk mencapai tujuan tersebut Selain dari aspek pengembangan teknologi dan akademis, penelitian ini juga memiliki dampak yang signifikan pada pendidikan dan pengembangan teknologi di bidang sistem kontrol. Prototipe dan temuan yang dilaporkan dapat bermanfaat bagi siswa, peneliti, dan industri sebagai sumber inspirasi, referensi, dan mungkin juga sebagai produk konkret di pasar.
\vspace*{1cm}

\noindent KATA KUNCI

Pemodelan, sistem dinamik, kontrol, MPC, Cart-Inverted Pendulum