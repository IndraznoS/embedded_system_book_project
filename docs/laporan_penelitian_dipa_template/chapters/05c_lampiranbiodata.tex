\chapter*{}
\addcontentsline{toc}{chapter}{Lampiran C. Biodata Tim Peneliti}
\vskip 7.5cm
\begin{center}
	\Large \textbf{Lampiran C.} Biodata Tim Peneliti
\end{center}
\newpage

\normalsize \textbf{Biodata Tim Pelaksana}\\
\begin{flushleft}
	\textbf{Jabatan dalam Tim Pengusul} (Ketua Peneliti)\\
	\textbf{A. Identitas Diri}
\end{flushleft}
\vskip -0.5cm
\begin{table}[h!]
	\resizebox{\textwidth}{!}{%
		\begin{tabular}{|l|l|l|}
			\hline
			1  & Nama Lengkap                                                         & \textbf{Indrazno Siradjuddin, ST., MT., PhD}   \\ \hline
			2  & Jenis Kelamin                                                        & Laki-laki                                      \\ \hline
			3  & Jabatan Fungsional                                                   & Lektor Kepala III/D                            \\ \hline
			4  & NIK                                                                  & 197406242000121001                             \\ \hline
			5  & NIDN                                                                 & 0024067406                                     \\ \hline
			6  & \begin{tabular}[c]{@{}l@{}}Tempat dan Tanggal\\ Lahir\end{tabular}   & Tulungagung, 24 Juni 1974                      \\ \hline
			7  & Alamat e-mail                                                        & {\color[HTML]{000000} indrazno@polinema.ac.id} \\ \hline
			8  & Nomor telepon/HP                                                     & 081222437474                                   \\ \hline
			9  & Alamat Kantor                                                        & JL Soekarno Hatta No. 9, Malang                  \\ \hline
			10 & \begin{tabular}[c]{@{}l@{}}Nomor Telepon/Fax\\ (Kantor)\end{tabular} & 0341404424, faks 0341404420                  \\ \hline
			11 & \begin{tabular}[c]{@{}l@{}}Mata Kuliah yang\\ diampu\end{tabular} & \begin{tabular}[c]{@{}l@{}}1. Sistem Embedded\\ 2. Robotika\\ 3. Robot Industri\end{tabular} \\ \hline
			12 & Scopus ID                                                            & {\color[HTML]{303030} 36633223300}             \\ \hline
		\end{tabular}%
	}
\end{table}
\vspace{-0.5cm}
\begin{flushleft}
	\textbf{B. Riwayat Pendidikan}
\end{flushleft}
\vspace{-0.5cm}
% Please add the following required packages to your document preamble:
% \usepackage{graphicx}
% \usepackage[normalem]{ulem}
% \useunder{\uline}{\ul}{}
\begin{table}[h!]
	\resizebox{\textwidth}{!}{%
		\begin{tabular}{|l|c|c|c|}
			\hline
			&
			S-1 &
			S-2 &
			S-3 \\ \hline
			\begin{tabular}[c]{@{}l@{}}Nama \\ Perguruan\\ Tinggi\end{tabular} &
			\begin{tabular}[c]{@{}c@{}}Universitas\\ Brawijaya\end{tabular} &
			ITS &
			\begin{tabular}[c]{@{}c@{}}Ulster \\ University\end{tabular} \\ \hline
			Bidang Ilmu &
			\begin{tabular}[c]{@{}c@{}}Teknik\\ Elektronika\end{tabular} &
			\begin{tabular}[c]{@{}c@{}}Teknik\\ Elektronika\end{tabular} &
			\begin{tabular}[c]{@{}c@{}}Intelligent \\ Systems\\ and Robotics\end{tabular} \\ \hline
			\begin{tabular}[c]{@{}l@{}}Tahun\\ Masuk-Lulus\end{tabular} &
			1993-2000 &
			2004-2006 &
			2009-2014 \\ \hline
			\begin{tabular}[c]{@{}l@{}}Judul Skripsi/\\ Tesis/Disertasi\end{tabular} &
			\multicolumn{1}{l|}{\begin{tabular}[c]{@{}l@{}}A Prototype of PC\\ Based Pipeline\\ Airflow Control\\ Using Fuzzy \\ Method\end{tabular}} &
			\multicolumn{1}{l|}{\begin{tabular}[c]{@{}l@{}}Indoor Mobile \\ Robot Exploration \\ and Navigation \\ using Probabilistic \\ Occupancy Grid \\ Maps\end{tabular}} &
			\multicolumn{1}{l|}{\begin{tabular}[c]{@{}l@{}}Computationally\\ Intelligent Visual\\ Servoing\end{tabular}} \\ \hline
			Pembimbing &
			\multicolumn{1}{l|}{\begin{tabular}[c]{@{}l@{}}Ir. Purwanto., MT.,\\ Ir. Chairuzaini\end{tabular}} &
			\multicolumn{1}{l|}{\begin{tabular}[c]{@{}l@{}}Ir. Joko Purwanto.,\\ M.Eng., PhD\\ Ir. Hendra Kusuma,\\ M.Eng\end{tabular}} &
			\multicolumn{1}{l|}{\begin{tabular}[c]{@{}l@{}}Prof. Laxmidhar\\ Bahera\\ Prof. Martin\\ McGinnity\\ Dr. Sonya \\ Coleman\end{tabular}} \\ \hline
		\end{tabular}%
	}
\end{table}
\newpage

\begin{flushleft}
	\textbf{C. Pengalaman Penelitian dalam 5 Tahun Terakhir}
\end{flushleft}
\vspace{-0.5cm}
% Please add the following required packages to your document preamble:
% \usepackage[normalem]{ulem}
% \useunder{\uline}{\ul}{}
% \usepackage{longtable}
% Note: It may be necessary to compile the document several times to get a multi-page table to line up properly
\begin{longtable}{|c|l|l|l|l|}
	\hline
	\textbf{No.} &
	\multicolumn{1}{c|}{\textbf{Tahun}} &
	\multicolumn{1}{c|}{\textbf{\begin{tabular}[c]{@{}c@{}}Judul\\ Penelitian\end{tabular}}} &
	\multicolumn{1}{c|}{\textbf{\begin{tabular}[c]{@{}c@{}}Sumber\\ Pendanaan\end{tabular}}} &
	\multicolumn{1}{c|}{\textbf{Jumlah}} \\ \hline
	\endhead
	%
	1 &
	\begin{tabular}[c]{@{}l@{}}2011-\\ 2012\end{tabular} &
	\begin{tabular}[c]{@{}l@{}}Intrinsically Motivated\\ Cumulative Learning \\ Versatile Robots \\ (Research Assistant)\end{tabular} &
	\begin{tabular}[c]{@{}l@{}}IM Clever\\ Project, \\ FP7-\\ ICT-IP-\\ 231722,\\ European\\ Union, \\ UK,\\ Th. 2011-\\ 2012\end{tabular} &
	\begin{tabular}[c]{@{}l@{}}7.726.783\\ (Euros)\end{tabular} \\ \hline
	2 &
	2015 &
	\begin{tabular}[c]{@{}l@{}}Sebagai Ketua dalam\\ melaksanakan Penelitian \\ Hibah Fundamental \\ dengan judul \\ "Pengembangan dan\\ Implementasi Sistem \\ Kontrol Visual Servoing \\ untuk Robot Beroda \\ Berjenis Differential\\ Drive"\end{tabular} &
	\begin{tabular}[c]{@{}l@{}}Hibah\\ Penelitian\\ DIKTI\end{tabular} &
	\begin{tabular}[c]{@{}l@{}}65.000.000\\ (IDR)\end{tabular} \\ \hline
	3 &
	2015 &
	\begin{tabular}[c]{@{}l@{}}Sebagai Ketua dalam\\ melaksanakan Penelitian \\ Hibah Fundamental \\ dengan judul \\ "Pengembangan dan\\ Implementasi Sistem \\ Kontrol Visual Servoing \\ untuk Robot Beroda \\ Berjenis Differential\\ Drive"\end{tabular} &
	\begin{tabular}[c]{@{}l@{}}Hibah\\ Penelitian\\ DIKTI\end{tabular} &
	\begin{tabular}[c]{@{}l@{}}65.000.000\\ (IDR)\end{tabular} \\ \hline
	4 &
	2015 &
	\begin{tabular}[c]{@{}l@{}}Sebagai Ketua dalam\\ melaksanakan Penelitian \\ DIPA Polinema dengan \\ Judul "Analisis dan \\ Implementasi Simulasi \\ Algoritma Sistem Kontrol\\ Visual Servoing pada \\ Beaglebone Black"\end{tabular} &
	\begin{tabular}[c]{@{}l@{}}Swadana\\ P2M\\ Polinema\end{tabular} &
	\begin{tabular}[c]{@{}l@{}}4.000.000\\ (IDR)\end{tabular} \\ \hline
	5 &
	2015 &
	\begin{tabular}[c]{@{}l@{}}Sebagai Anggota dalam\\ melaksanakan Penelitian \\ DIPA Polinema dengan \\ Judul "Analisa Performansi \\ Sensor Linear Hall Efek \\ pada Detektor Arus Beban \\ Transformator Step Down"\end{tabular} &
	\begin{tabular}[c]{@{}l@{}}Swadana\\ P2M\\ Polinema\end{tabular} &
	\begin{tabular}[c]{@{}l@{}}4.000.000\\ (IDR)\end{tabular} \\ \hline
	6 &
	2016 &
	\begin{tabular}[c]{@{}l@{}}Sebagai Ketua dalam \\ Penelitian Hibah \\ Fundamental dengan \\ Judul" Pemodelan, Analisis \\ dan Implementasi \\ Sistem Kontrol Alat \\ Transportasi Pendulum \\ Terbalik dengan 2 Roda \\ Penggerak"\end{tabular} &
	\begin{tabular}[c]{@{}l@{}}Hibah\\ Penelitian\\ DIKTI\end{tabular} &
	\begin{tabular}[c]{@{}l@{}}60.000.000\\ (IDR)\end{tabular} \\ \hline
	7 &
	2016 &
	\begin{tabular}[c]{@{}l@{}}Sebagai Ketua dalam \\ Penelitian Hibah \\ Fundamental  dengan \\ Judul" Pemodelan, \\ Analisis dan Implementasi \\ Sistem Kontrol Alat \\ Transportasi Pendulum \\ Terbalik dengan 2 Roda \\ Penggerak"\end{tabular} &
	\begin{tabular}[c]{@{}l@{}}Hibah\\ Penelitian\\ DIKTI\end{tabular} &
	\begin{tabular}[c]{@{}l@{}}60.000.000\\ (IDR)\end{tabular} \\ \hline
	8 &
	\begin{tabular}[c]{@{}l@{}}2018-\\ 2020\end{tabular} &
	\begin{tabular}[c]{@{}l@{}}Sebagai Ketua dalam \\ Penelitian Pengembangan \\ dan Implementasi Algoritma \\ dan Teknik Kendali Robot \\ Swatantra: Pelokalisasian\\ Diri, Pengkonstruksian Peta \\ dan Perencanaan Gerak\end{tabular} &
	\begin{tabular}[c]{@{}l@{}}Hibah\\ Penelitian\\ DIKTI\end{tabular} &
	\begin{tabular}[c]{@{}l@{}}400.000.000\\ (IDR)\end{tabular} \\ \hline
	9 &
	2018 &
	\begin{tabular}[c]{@{}l@{}}Sebagai Anggota Kegiatan\\ Penelitian Sistem Monitoring\\ Trafo Gardu Distribusi \\ PT. PLN (Persero) Berbasis \\ IOT\end{tabular} &
	\begin{tabular}[c]{@{}l@{}}Swadana\\ P2M\\ Polinema\end{tabular} &
	\begin{tabular}[c]{@{}l@{}}190.000.000\\ (IDR)\end{tabular} \\ \hline
	10 &
	2018 &
	\begin{tabular}[c]{@{}l@{}}Sebagai Anggota Kegiatan\\ Penelitian Disain Dan \\ Analisis Plannar Sleeve \\ Antenna Untuk Sistem \\ Televisi Dan Sistem Kontrol, \\ Navigasi Robot Pada\\ Lingkungan Dinamis\end{tabular} &
	\begin{tabular}[c]{@{}l@{}}Swadana\\ P2M\\ Polinema\end{tabular} &
	\begin{tabular}[c]{@{}l@{}}90.000.000\\ (IDR)\end{tabular} \\ \hline
\end{longtable}

\begin{flushleft}
	\textbf{D. Pengalaman Pengabdian pada Masyarakat dalam 5 Tahun Terakhir}
\end{flushleft}
\vspace{-0.5cm}
% Please add the following required packages to your document preamble:
% \usepackage{graphicx}
% \usepackage[normalem]{ulem}
% \useunder{\uline}{\ul}{}
\begin{table}[h!]
	\resizebox{\textwidth}{!}{%
		\begin{tabular}{|c|l|l|l|}
			\hline
			\textbf{No.} &
			\multicolumn{1}{c|}{\textbf{Tahun}} &
			\multicolumn{1}{c|}{\textbf{\begin{tabular}[c]{@{}c@{}}Judul Pengabdian \\ Kepada  Masyarakat\end{tabular}}} &
			\multicolumn{1}{c|}{\textbf{\begin{tabular}[c]{@{}c@{}}Sumber\\ Pendanaan\end{tabular}}} \\ \hline
			1 &
			2014-2015 &
			\begin{tabular}[c]{@{}l@{}}Sebagai Anggota dalam melaksanakan \\ Pengabdian "Perakitan dan Instalasi Bel \\ Sekolah MAN 1 Malang Menggunakan \\ KIT BS13 serta Pelatihan Cara\\ Pengoperasiannya"\end{tabular} &
			\begin{tabular}[c]{@{}l@{}}Swadana \\ P2M\\ Polinema\end{tabular} \\ \hline
			2 &
			2014-2015 &
			\begin{tabular}[c]{@{}l@{}}Sebagai anggota dalam melaksanakan \\ Pengabdian "Perakitan dan Pelatihan \\ Jam Pengingat Waktu Sholat di Masjid \\ Baitul Mukminin dusun Biru Desa\\ Gunungrejo, Singosari."\end{tabular} &
			\begin{tabular}[c]{@{}l@{}}Swadana \\ P2M\\ Polinema\end{tabular} \\ \hline
		\end{tabular}%
	}
\end{table}

\begin{flushleft}
	\textbf{E. Publikasi Artikel Ilmiah dalam Jurnal 5 Tahun Terakhir}
\end{flushleft}
\vspace{-0.5cm}
% Please add the following required packages to your document preamble:
% \usepackage[table,xcdraw]{xcolor}
% If you use beamer only pass "xcolor=table" option, i.e. \documentclass[xcolor=table]{beamer}
% \usepackage{longtable}
% Note: It may be necessary to compile the document several times to get a multi-page table to line up properly
\begin{longtable}{|c|l|l|l|}
	\hline
	\textbf{No.} &
	\multicolumn{1}{c|}{\textbf{\begin{tabular}[c]{@{}c@{}}Judul\\ Artikel Ilmiah\end{tabular}}} &
	\multicolumn{1}{c|}{\textbf{Nama Jurnal}} &
	\multicolumn{1}{c|}{\textbf{\begin{tabular}[c]{@{}c@{}}Volume/\\ Nomor/\\ Tahun\end{tabular}}} \\ \hline
	\endhead
	%
	1 &
	\begin{tabular}[c]{@{}l@{}}An Iterative Robot-Image \\ Jacobian Approximation of \\ Image-Based Visual Servoing \\ for Joint Limit Avoidance.\end{tabular} &
	\begin{tabular}[c]{@{}l@{}}International\\ Journal of\\ Mechatronics\\ and Automation\end{tabular} &
	\begin{tabular}[c]{@{}l@{}}Vol.2, No.2,\\ 227-239, 2012\\ pp\end{tabular} \\ \hline
	2 &
	\begin{tabular}[c]{@{}l@{}}Image-Based Visual Servoing \\ of a 7 DOF Robot Manipulator \\ Using an Adaptive Distributed \\ Fuzzy PD Controller.\end{tabular} &
	\begin{tabular}[c]{@{}l@{}}IEEE \\ Transactions\\ on \\ Mechatronics\end{tabular} &
	\begin{tabular}[c]{@{}l@{}}Vol.19, No.2,\\ 512-523, 2014\\ pp\end{tabular} \\ \hline
	3 &
	\begin{tabular}[c]{@{}l@{}}An Image Based Visual Control \\ Law for a Differential Drive \\ Mobile Robot\end{tabular} &
	\begin{tabular}[c]{@{}l@{}}International\\ Journal of\\ Mechanical \&\\ Mechatronics\\ Engineering\\ IJMME-IJENS\end{tabular} &
	\begin{tabular}[c]{@{}l@{}}Vol:15 NO:06,\\ 100-107, 2015\\ pp\end{tabular} \\ \hline
	4 &
	\begin{tabular}[c]{@{}l@{}}Perancangan dan Implementasi \\ Sistem Kontrol Differensial \\ Drive Personal Transporter \\ dengan Menggunakan Sistem \\ Kontrol PID\end{tabular} &
	\begin{tabular}[c]{@{}l@{}}Jurnal \\ Elektronika\\ Otomasi \\ Industri\end{tabular} &
	\begin{tabular}[c]{@{}l@{}}Vol 2, No. 1 \\ pp 2-8, 2015\end{tabular} \\ \hline
	5 &
	\begin{tabular}[c]{@{}l@{}}Pengaturan Kecepatan Motor \\ DC Terintegrasi Untuk Sistem \\ Penggerak Directional Robot \\ Dengan Metode PID\end{tabular} &
	\begin{tabular}[c]{@{}l@{}}Jurnal \\ Elektronika\\ Otomasi \\ Industri\end{tabular} &
	\begin{tabular}[c]{@{}l@{}}Vol 2, No. 1 \\ pp 54-59, \\ 2015\end{tabular} \\ \hline
	6 &
	\begin{tabular}[c]{@{}l@{}}Pemodelan dan Analisis Sistem \\ Kontrol Kinematik Omni \\ Directional pada Robot\end{tabular} &
	\begin{tabular}[c]{@{}l@{}}Jurnal \\ Elektronika\\ Otomasi \\ Industri\end{tabular} &
	\begin{tabular}[c]{@{}l@{}}Vol 2, No. 1 \\ pp 80-86, \\ 2015\end{tabular} \\ \hline
	7 &
	\begin{tabular}[c]{@{}l@{}}State space control using LQR \\ method for a cart-inverted \\ pendulum linearised model\end{tabular} &
	\begin{tabular}[c]{@{}l@{}}International\\ Journal of\\ Mechanical \&\\ Mechatronics\\ Engineering\\ IJMME-IJENS\end{tabular} &
	\begin{tabular}[c]{@{}l@{}}Vol 17, No. 1 \\ pp 119-126, \\ 2017\end{tabular} \\ \hline
	8 &
	\begin{tabular}[c]{@{}l@{}}A New Implementation Of \\ Single Phase Shimizu Inverter \\ For Optimal Power Flow Of \\ Solar PV System Based On \\ Incremental Conductance \\ MPPT Method\end{tabular} &
	{\color[HTML]{37393B} \begin{tabular}[c]{@{}l@{}}ICIC Express\\ Letters\end{tabular}} &
	\begin{tabular}[c]{@{}l@{}}Tahun: 2017 \\ Volume: 12 \\ ISSN: 1881-\\ 803X\end{tabular} \\ \hline
	9 &
	\begin{tabular}[c]{@{}l@{}}The Development of \\ Classification System of \\ Student Final Assignment  \\ Using Naive Bayes \\ Classifier Case Study: State \\ Community Academy \\ of Bojonegoro\end{tabular} &
	\begin{tabular}[c]{@{}l@{}}International\\ Journal of\\ Engineering\\ and Technology \\ (UAE)\end{tabular} &
	{\color[HTML]{37393B} \begin{tabular}[c]{@{}l@{}}Tahun: 2018\\ Volume: 7\\ ISSN: \\ 2227-524X\end{tabular}} \\ \hline
	10 &
	\begin{tabular}[c]{@{}l@{}}State-feedback control with a \\ full-state estimator for a \\ cart-inverted pendulum system\end{tabular} &
	{\color[HTML]{37393B} \begin{tabular}[c]{@{}l@{}}International\\ Journal of\\ Engineering\\ and Technology \\ (UAE)\end{tabular}} &
	{\color[HTML]{37393B} \begin{tabular}[c]{@{}l@{}}Tahun: 2018\\ Volume: 7\\ ISSN: \\ 2227-524X\end{tabular}} \\ \hline
	11 &
	\begin{tabular}[c]{@{}l@{}}Raspberry Pi-Based Farming \\ Automation and Monitoring \\ System using Automatic \\ Weather System \\ (AWS) \\ (Case Study: Chili Plants)\end{tabular} &
	{\color[HTML]{37393B} \begin{tabular}[c]{@{}l@{}}International\\ Journal of\\ Engineering\\ and Technology \\ (UAE)\end{tabular}} &
	{\color[HTML]{37393B} \begin{tabular}[c]{@{}l@{}}Tahun: 2018\\ Volume: 7\\ ISSN: \\ 2227-524X\end{tabular}} \\ \hline
	12 &
	\begin{tabular}[c]{@{}l@{}}Open Problems in Indonesian \\ Automatic Essay Scoring System\end{tabular} &
	{\color[HTML]{37393B} \begin{tabular}[c]{@{}l@{}}International\\ Journal of\\ Engineering\\ and Technology \\ (UAE)\end{tabular}} &
	{\color[HTML]{37393B} \begin{tabular}[c]{@{}l@{}}Tahun: 2018\\ Volume: 7\\ ISSN: \\ 2227-524X\end{tabular}} \\ \hline
	13 &
	\begin{tabular}[c]{@{}l@{}}Geo-Sentiment Analysis as a \\ Location-Based Opinion \\ Analysis  System on Public\\ Opinion Data about Governor \\ Candidates\end{tabular} &
	{\color[HTML]{37393B} \begin{tabular}[c]{@{}l@{}}International\\ Journal of\\ Engineering\\ and Technology \\ (UAE)\end{tabular}} &
	{\color[HTML]{37393B} \begin{tabular}[c]{@{}l@{}}Tahun: 2018\\ Volume: 7\\ ISSN: \\ 2227-524X\end{tabular}} \\ \hline
	14 &
	\begin{tabular}[c]{@{}l@{}}Design and Analysis of Ultra \\ Low-Profile ILA on \\ a Rectangular Conducting \\ Plane\end{tabular} &
	{\color[HTML]{37393B} \begin{tabular}[c]{@{}l@{}}International\\ Journal of\\ Engineering\\ and Technology \\ (UAE)\end{tabular}} &
	{\color[HTML]{37393B} \begin{tabular}[c]{@{}l@{}}Tahun: 2018\\ Volume: 7\\ ISSN: \\ 2227-524X\end{tabular}} \\ \hline
	15 &
	\begin{tabular}[c]{@{}l@{}}A New Implementation of \\ Single  Phase Shimizu Inverter \\ for Optimal  Power Flow \\ of Solar PV System Based \\ on Incremental Conductance \\ MPPT Method\end{tabular} &
	{\color[HTML]{37393B} \begin{tabular}[c]{@{}l@{}}International\\ Journal of\\ Research and\\ Surveys\end{tabular}} &
	{\color[HTML]{37393B} \begin{tabular}[c]{@{}l@{}}Tahun: 2018\\ Volume: 12\\ ISSN: \\ 1881-803X\end{tabular}} \\ \hline
	16 &
	\begin{tabular}[c]{@{}l@{}}Kinematics and Control \\ A Three Wheeled \\ Omnidirectional Mobile \\ Robot\end{tabular} &
	{\color[HTML]{37393B} \begin{tabular}[c]{@{}l@{}}SSRG\\ International\\ Journal of\\ Electrical and\\ Electronics\\ Engineering\end{tabular}} &
	{\color[HTML]{37393B} \begin{tabular}[c]{@{}l@{}}Tahun: 2019\\ Volume 6\\ Issue 12\\ E-ISSN \\ 2348 - 8379\end{tabular}} \\ \hline
	17 &
	\begin{tabular}[c]{@{}l@{}}Stabilized controller of a two \\ wheels robot\end{tabular} &
	{\color[HTML]{37393B} \begin{tabular}[c]{@{}l@{}}Bulletin of\\ Electrical\\ Engineering \\ and\\ Informatics\end{tabular}} &
	{\color[HTML]{37393B} \begin{tabular}[c]{@{}l@{}}Tahun : 2020\\ Volume 9\\ Issue 6\\ ISSN:\\ 2302-9285\end{tabular}} \\ \hline
	18 &
	\begin{tabular}[c]{@{}l@{}}Linear quadratic regulator \\ and pole placement for \\ stabilizing  a cart inverted \\ pendulum system\end{tabular} &
	{\color[HTML]{37393B} \begin{tabular}[c]{@{}l@{}}Bulletin of\\ Electrical\\ Engineering \\ and\\ Informatics\end{tabular}} &
	{\color[HTML]{37393B} \begin{tabular}[c]{@{}l@{}}Tahun : 2020\\ Volume 9\\ Issue 6\\ ISSN:\\ 2302-9285\end{tabular}} \\ \hline
\end{longtable}

\begin{flushleft}
	\textbf{F. Pemakalah Seminar Ilmiah (\textit{oral presentation}) dalam 5 Tahun Terakhir}
\end{flushleft}
\vspace{-0.5cm}
% Please add the following required packages to your document preamble:
% \usepackage{longtable}
% Note: It may be necessary to compile the document several times to get a multi-page table to line up properly
\begin{longtable}{|l|l|l|l|}
	\hline
	\multicolumn{1}{|c|}{\textbf{No}} &
	\multicolumn{1}{c|}{\textbf{\begin{tabular}[c]{@{}c@{}}Nama Pertemuan\\ Ilmiah/Seminar\end{tabular}}} &
	\multicolumn{1}{c|}{\textbf{\begin{tabular}[c]{@{}c@{}}Judul\\ Artikel\\ Imiah\end{tabular}}} &
	\multicolumn{1}{c|}{\textbf{\begin{tabular}[c]{@{}c@{}}Waktu\\ dan\\ Tempat\end{tabular}}} \\ \hline
	\endhead
	%
	\multicolumn{1}{|c|}{1} &
	\begin{tabular}[c]{@{}l@{}}IEEE IET Irish Signals \\ and Systems Conference\end{tabular} &
	\begin{tabular}[c]{@{}l@{}}Visual Servoing of a \\ Redundant Manipulator \\ Using Shape Moments\end{tabular} &
	\begin{tabular}[c]{@{}l@{}}Dublin, \\ Juni,\\ 2009\end{tabular} \\ \hline
	\multicolumn{1}{|c|}{2} &
	\begin{tabular}[c]{@{}l@{}}IEEE International\\ Conference on Fuzzy \\ Systems\end{tabular} &
	\begin{tabular}[c]{@{}l@{}}Image-Based Visual \\ Servoing of a 7 DOF \\ Robot Manipulator \\ Using a Distributed \\ Fuzzy Proportional\\ Controller\end{tabular} &
	\begin{tabular}[c]{@{}l@{}}Barcelona, \\ Juli 2010\end{tabular} \\ \hline
	\multicolumn{1}{|c|}{3} &
	\begin{tabular}[c]{@{}l@{}}IEEE International\\ Conference on \\ Mechatronics\\ and Automation\end{tabular} &
	\begin{tabular}[c]{@{}l@{}}A Computationally \\ Efficient Approach\\ for Jacobian \\ Approximation for \\ Joint Limit Avoidance\end{tabular} &
	\begin{tabular}[c]{@{}l@{}}Beijing, \\ 2011\end{tabular} \\ \hline
	\multicolumn{1}{|c|}{4} &
	\begin{tabular}[c]{@{}l@{}}IEEE International Joint \\ Conference on Neural \\ Networks\end{tabular} &
	\begin{tabular}[c]{@{}l@{}}A Position Based Visual \\ Tracking System for \\ a 7 DOF Robot \\ Manipulator Using \\ a Kinect Camera\end{tabular} &
	\begin{tabular}[c]{@{}l@{}}Sydney, \\ Juni 2012\end{tabular} \\ \hline
	\multicolumn{1}{|c|}{5} &
	\begin{tabular}[c]{@{}l@{}}IEEE International\\ Conference on Robotics \\ and Biomimetics\end{tabular} &
	\begin{tabular}[c]{@{}l@{}}Primitive Action \\ Learning using Fuzzy\\ Neural Networks\end{tabular} &
	\begin{tabular}[c]{@{}l@{}}Guangzhou,\\ Desember \\ 2012\end{tabular} \\ \hline
	\multicolumn{1}{|c|}{6} &
	\begin{tabular}[c]{@{}l@{}}International \\ Conference on Vocational \\ Education \\ and Electrical \\ Engineering\end{tabular} &
	\begin{tabular}[c]{@{}l@{}}Modelling and Analysis \\ of a Photovoltaic Cell\end{tabular} &
	\begin{tabular}[c]{@{}l@{}}Surabaya,\\ November\\ 2015\end{tabular} \\ \hline
	\multicolumn{1}{|c|}{7} &
	\begin{tabular}[c]{@{}l@{}}IEEE International\\ Symposium on Robotics \\ and Intelligent Sensors\end{tabular} &
	\begin{tabular}[c]{@{}l@{}}A Real-Time Model \\ Based Visual Servoing \\ Application for \\ a Differential Drive \\ Mobile Robot Using \\ Beaglebone Black \\ Embedded System\end{tabular} &
	\begin{tabular}[c]{@{}l@{}}Langkawi,\\ Oktober \\ 2015\end{tabular} \\ \hline
	8 &
	\begin{tabular}[c]{@{}l@{}}International \\ Symposium \\ On Novel And \\ Sustainable Technology\end{tabular} &
	\begin{tabular}[c]{@{}l@{}}One-shot learning \\ Algorithm for Imitating \\ Primitive Movements \\ of a Robot Actuator\end{tabular} &
	\begin{tabular}[c]{@{}l@{}}Tainan,\\ Oktober \\ 2016\end{tabular} \\ \hline
	9 &
	\begin{tabular}[c]{@{}l@{}}Quality in Research, \\ Indexed by \\ IEEE Explore\end{tabular} &
	\begin{tabular}[c]{@{}l@{}}Stabilising A Cart \\ Inverted Pendulum\\ System Using Pole \\ Placement Control\\ Method (Accepted, \\ It will be presented)\end{tabular} &
	\begin{tabular}[c]{@{}l@{}}Bali, July \\ 2017\end{tabular} \\ \hline
	10 &
	\begin{tabular}[c]{@{}l@{}}Quality in Research, \\ Indexed by \\ IEEE Explore\end{tabular} &
	\begin{tabular}[c]{@{}l@{}}Identification of \\ Pulse Frequency\\ Spectrum of Chronic \\ Kidney Disease\\ Patients Measured at \\ TCM Points\\ Using FFT Processing \\ (Accepted, It\\ will be presented)\end{tabular} &
	\begin{tabular}[c]{@{}l@{}}Bali, July \\ 2018\end{tabular} \\ \hline
	11 &
	\begin{tabular}[c]{@{}l@{}}International Conference \\ on Applied Science and\\ Technology (iCAST on\\ Engineering Science)\end{tabular} &
	\begin{tabular}[c]{@{}l@{}}Cluster implementation \\ on mini Raspberry Pi \\ computers using Round\\ Robin Algorithm\end{tabular} &
	\begin{tabular}[c]{@{}l@{}}Bali, \\ Oktober\\ 2019\end{tabular} \\ \hline
	12 &
	AASEC &
	\begin{tabular}[c]{@{}l@{}}PID controller for \\ a differential drive\\ robot balancing system\end{tabular} &
	\begin{tabular}[c]{@{}l@{}}Bandung, \\ April\\ 2019\end{tabular} \\ \hline
	13 &
	AASEC &
	\begin{tabular}[c]{@{}l@{}}A study of a discrete \\ Bayes and a Kalman filter \\ computational Complexity\\ and performance in the \\ case of 1D robot\\ localization\end{tabular} &
	\begin{tabular}[c]{@{}l@{}}Bandung, \\ April\\ 2019\end{tabular} \\ \hline
	14 &
	AASEC &
	\begin{tabular}[c]{@{}l@{}}The study of the wideband \\ planar sleeve antenna\end{tabular} &
	\begin{tabular}[c]{@{}l@{}}Bandung, \\ April\\ 2019\end{tabular} \\ \hline
	15 &
	AASEC &
	\begin{tabular}[c]{@{}l@{}}Study of LoRa (Long \\ Range) communication \\ for monitoring of a ship \\ electrical system\end{tabular} &
	\begin{tabular}[c]{@{}l@{}}Bandung, \\ April\\ 2019\end{tabular} \\ \hline
	16 &
	AASEC &
	\begin{tabular}[c]{@{}l@{}}An improved control \\ method  to reduce \\ harmonic level for \\ a single phase \\ grid connected flyback \\ micro-inverter of a\\ small scale solar PV\end{tabular} &
	\begin{tabular}[c]{@{}l@{}}Bandung, \\ April\\ 2019\end{tabular} \\ \hline
	17 &
	AASEC &
	\begin{tabular}[c]{@{}l@{}}Glidding system for \\ a fixed wing aircraft \\ using PID control \\ algorithm\end{tabular} &
	\begin{tabular}[c]{@{}l@{}}Bandung, \\ April\\ 2019\end{tabular} \\ \hline
	18 &
	AASEC &
	\begin{tabular}[c]{@{}l@{}}Designing, implementing \\ and analysing optimal \\ controllers on a non-linear\\ reaction wheel pendulum\end{tabular} &
	\begin{tabular}[c]{@{}l@{}}Bandung, \\ April\\ 2019\end{tabular} \\ \hline
	19 &
	\begin{tabular}[c]{@{}l@{}}IEEE Conference on \\ Antenna Measurements \\ \& Applications\\ (CAMA)\end{tabular} &
	\begin{tabular}[c]{@{}l@{}}Study of Parasitic Element \\ Effects of Multiband IFA \\ for UHF and SHF\\ Channel Systems\end{tabular} &
	Bali, 2019 \\ \hline
	20 &
	\begin{tabular}[c]{@{}l@{}}International Conference \\ on Information and\\ Communications \\ Technology (ICOIACT)\end{tabular} &
	\begin{tabular}[c]{@{}l@{}}A Power Sharing Loop \\ Control Method for \\ Input-series \\ Output-parallel \\ Flyback type \\ Micro-Inverter \\ Using Droop Method\end{tabular} &
	\begin{tabular}[c]{@{}l@{}}Jogyakarta, \\ 2019\end{tabular} \\ \hline
\end{longtable}

\begin{flushleft}
	\textbf{G. Karya Buku dalam 5 Tahun Terakhir}
\end{flushleft}
\vspace{-0.5cm}
% Please add the following required packages to your document preamble:
% \usepackage[normalem]{ulem}
% \useunder{\uline}{\ul}{}
% \usepackage{longtable}
% Note: It may be necessary to compile the document several times to get a multi-page table to line up properly
\begin{longtable}{|l|l|l|l|}
	\hline
	\multicolumn{1}{|c|}{\textbf{No}} &
	\multicolumn{1}{c|}{\textbf{Judul Buku}} &
	\multicolumn{1}{c|}{\textbf{Penerbit}} &
	\multicolumn{1}{c|}{\textbf{Tahun}} \\ \hline
	\endhead
	%
	1 &
	\begin{tabular}[c]{@{}l@{}}Embedded System Berbasiskan Beaglebone \\ Black (ISBN: 978-602-19379-8-3)\end{tabular} &
	Polinema Press &
	2016 \\ \hline
\end{longtable}

\begin{flushleft}
	\textbf{H. Perolehan HKI dalam 5-10 Tahun Terakhir}
\end{flushleft}
\vspace{-0.5cm}
% Please add the following required packages to your document preamble:
% \usepackage{longtable}
% Note: It may be necessary to compile the document several times to get a multi-page table to line up properly
\begin{longtable}{|c|l|l|l|l|}
	\hline
	\textbf{No} &
	\multicolumn{1}{c|}{\textbf{Judul/Tema HKI}} &
	\multicolumn{1}{c|}{\textbf{Tahun}} &
	\multicolumn{1}{c|}{\textbf{Jenis}} &
	\multicolumn{1}{c|}{\textbf{\begin{tabular}[c]{@{}c@{}}Nomor \\ P/ID\end{tabular}}} \\ \hline
	\endhead
	%
	1 & .................................................... & .......... & .................. & ................ \\ \hline
	2 & .................................................... & .......... & .................. & ................ \\ \hline
	3 & .................................................... & .......... & .................. & ................ \\ \hline
\end{longtable}

\begin{flushleft}
	\textbf{I. Pengalaman merumuskan Kebijakan Publik/Rekayasa Sosial Lainnya dalam 5 Tahun Terakhir}
\end{flushleft}
\vspace{-0.5cm}
% Please add the following required packages to your document preamble:
% \usepackage{longtable}
% Note: It may be necessary to compile the document several times to get a multi-page table to line up properly
\begin{longtable}{|l|l|l|l|}
	\hline
	\multicolumn{1}{|c|}{\textbf{No}} &
	\multicolumn{1}{c|}{\textbf{Jenis Penghargaan}} &
	\multicolumn{1}{c|}{\textbf{\begin{tabular}[c]{@{}c@{}}Institusi\\ Pemberi\\ Penghargaan\end{tabular}}} &
	\multicolumn{1}{c|}{\textbf{Tahun}} \\ \hline
	\endhead
	%
	1 &
	Visiting Lecturer &
	\begin{tabular}[c]{@{}l@{}}Southern Taiwan\\ University of Science\\ and Technology\end{tabular} &
	\begin{tabular}[c]{@{}l@{}}26 April – \\ 1 Mei 2015\end{tabular} \\ \hline
	2 &
	\begin{tabular}[c]{@{}l@{}}IEEE-RAS International \\ Robot PRIDE Compitition\end{tabular} &
	IEEE-RAS Malaysia &
	\begin{tabular}[c]{@{}l@{}}17-18 Oktober \\ 2015\end{tabular} \\ \hline
	3 &
	\begin{tabular}[c]{@{}l@{}}Penyaji terbaik pada \\ seminar hasil penelitian \\ kompetitif nasional (2015)\\ skema fundamental\end{tabular} &
	Kemenristekdikti &
	2016 \\ \hline
	4 &
	\begin{tabular}[c]{@{}l@{}}Invited Speaker \\ ”International Symposium \\ on Novel and Sustainable\\ Technology”\end{tabular} &
	\begin{tabular}[c]{@{}l@{}}Southern Taiwan\\ University of Science\\ and Technology\end{tabular} &
	\begin{tabular}[c]{@{}l@{}}6-7 Oktober \\ 2016\end{tabular} \\ \hline
	5 &
	\begin{tabular}[c]{@{}l@{}}Invited Speaker \\ “Seminar Nasional\\ Terapan Riset Inovatif \\ (SENTRINOV)”\end{tabular} &
	\begin{tabular}[c]{@{}l@{}}Politeknik Negeri\\ Malang\end{tabular} &
	\begin{tabular}[c]{@{}l@{}}23-24 November \\ 2017\end{tabular} \\ \hline
\end{longtable}

%\begin{flushleft}
%	\textbf{J. Penghargaan dalam 10 tahun terakhir(dari %Pemerintah, Asosiasi atau Institusi lainnya)}
%\end{flushleft}
%\vspace{5cm}
\pagebreak
\justify
Semua data yang saya isikan dan tercantum dalam biodata ini adalah benar dan dapat dipertanggung jawabkan secara hukum. Apabila dikemudian hari ternyata dijumpai ketidak sesuaian dengan kenyataan, saya sanggup menerima sanksi.\\
Demikian biodata ini saya buat dengan sebenarnya untuk memenuhi salah satu persyaratan dalam pengajuan Penelitian skema RISET TERAPAN dengan judul "Aplikasi Software Prediksi dan Kontrol Infeksi Virus Covid19".\\
\vskip 1cm
\begin{flushleft}
	\hspace{7cm} Malang, 6 Nopember 2022 \\
	\hspace{7cm} Ketua Tim Pengusul\\
	\vskip 2cm
	\hspace{7cm} Indrazno Siradjuddin, ST., MT., PhD\\
	\hspace{7cm} NIP. 196404091994031002
\end{flushleft}
