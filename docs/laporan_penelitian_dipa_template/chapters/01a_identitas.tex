%Identitas dan Uraian Umum Penelitian
\addcontentsline{toc}{chapter}{Identitas dan Uraian Umum Penelitian}
\title{Identitas}
\begin{center}
	\large \textbf{IDENTITAS DAN URAIAN UMUM PENELITIAN}
\end{center}
\vspace{0.5cm}
\textbf{1. Judul Penelitian:}\\
{DISAIN DAN IMPLEMENTASI PERANGKAT PEMBELAJARAN SISTEM OTOMASI INDUSTRI DENGAN MODEL PREDICTIVE CONTROL: STUDI KASUS CART-INVERTED PENDULUM} \\

\hspace{-0.6cm}\textbf{2. Tim Peneliti:}


\begin{table}[h!]
	\centering
	\resizebox{\textwidth}{!}{%
		\begin{tabular}{|c|l|c|c|c|c|}
			\hline
			\textbf{No}&
			\multicolumn{1}{c|}{\textbf{Nama}} &
			\textbf{Jabatan}&
			\textbf{\begin{tabular}[c]{@{}c@{}}Bidang \\ Keahlian\end{tabular}}&
			\textbf{Jurusan}&
			\textbf{\begin{tabular}[c]{@{}c@{}}Alokasi waktu\\ (Jam/Minggu)\end{tabular}}\\ \hline
			1 &
			\begin{tabular}[c]{@{}l@{}}Indrazno Siradjuddin, \\ S.T., M.T., Ph.D.\end{tabular} &
			\begin{tabular}[c]{@{}c@{}}Ketua Peneliti\\ Dosen \\ DIV-PSTE\end{tabular} &
			Robotika &
			\begin{tabular}[c]{@{}c@{}}Teknik\\ Elektro\end{tabular} &
			10 \\ \hline
			2 &
			\begin{tabular}[c]{@{}l@{}}Prof. Dr. Budhy Setiawan, B.SEET., MT, \end{tabular} &
			\begin{tabular}[c]{@{}c@{}}Anggota Peneliti \\ Dosen \\ DIV-PSTE\end{tabular} &
			\begin{tabular}[c]{@{}c@{}}Mekatronika\end{tabular} &
			\begin{tabular}[c]{@{}c@{}}Teknik\\ Elektro \end{tabular} &
			10 \\ \hline
			3 &
			\begin{tabular}[c]{@{}l@{}}Gillang Al Azhar, S.S.T., M.Tr.T \end{tabular} &
			\begin{tabular}[c]{@{}c@{}}Anggota Peneliti \\ Dosen \\ DIII-PSTE \end{tabular} &
			\begin{tabular}[c]{@{}c@{}}Robotika \end{tabular} &
			\begin{tabular}[c]{@{}c@{}}Teknik\\ Elektro \end{tabular} &
			10 \\ \hline
			4 &			
			\begin{tabular}[c]{@{}l@{}}Zakiyah Amalia, ST., M.Tr.T\end{tabular} &
			\begin{tabular}[c]{@{}c@{}}Pembantu Peneliti \\ Dosen \\ DIV-PSTM\end{tabular} &
			\begin{tabular}[c]{@{}c@{}}Sistem Kontrol \end{tabular} &
			\begin{tabular}[c]{@{}c@{}}Teknik\\ Mesin\end{tabular} &
			10 \\ \hline
			5 &			
			\begin{tabular}[c]{@{}l@{}}Ida Lailatul Fitria \end{tabular} &
			\begin{tabular}[c]{@{}c@{}}Mahasiswa \\ MTTE \end{tabular} &
			\begin{tabular}[c]{@{}c@{}}Sistem kontrol \\ dan Robotika\end{tabular} &
			\begin{tabular}[c]{@{}c@{}}Teknik\\ Elektro\end{tabular} &
			10 \\ \hline
			6&			
			\begin{tabular}[c]{@{}l@{}}Febby Ayu Salsabillah \end{tabular} &
			\begin{tabular}[c]{@{}c@{}}Mahasiswa \\ MTTE  \end{tabular} &
			\begin{tabular}[c]{@{}c@{}}Sistem kontrol \\ dan Robotika\end{tabular} &
			\begin{tabular}[c]{@{}c@{}}Teknik\\ Elektro\end{tabular} &
			10 \\ \hline
			8&			
			\begin{tabular}[c]{@{}l@{}}Dimas Adi Prayoga \end{tabular} &
			\begin{tabular}[c]{@{}c@{}}Mahasiswa \\ MTTE \end{tabular} &
			\begin{tabular}[c]{@{}c@{}}Sistem kontrol \\ dan Robotika\end{tabular} &
			\begin{tabular}[c]{@{}c@{}}Teknik\\ Elektro\end{tabular} &
			10 \\ \hline
            9&			
			\begin{tabular}[c]{@{}l@{}}Arif Anwar Rosyidin \end{tabular} &
			\begin{tabular}[c]{@{}c@{}}Mahasiswa \\ MTTE \end{tabular} &
			\begin{tabular}[c]{@{}c@{}}Sistem kontrol \\ dan Robotika\end{tabular} &
			\begin{tabular}[c]{@{}c@{}}Teknik\\ Elektro\end{tabular} &
			10 \\ \hline
		\end{tabular}%
	}
\end{table}


\hspace{-0.6cm}\textbf{3. Objek Penelitian (jenis material yang akan diteliti dan segi penelitian):}\\
Obyek penelitian adalah produk cart-inverted pendulum yang dilengkapi dengan sensor, embedded system serta software untuk implementasi MPC secara realtime \\

\hspace{-0.6cm}\textbf{4. Masa Pelaksanaan:}\\
8 bulan\\

\hspace{-0.6cm}\textbf{5. Biaya:}\\
Rp 40.000.000,00\\

\hspace{-0.6cm}\textbf{6. Lokasi Penelitian:}\\
Laboratorium MTTE dan Workshop di tempat Mitra \\

\hspace{-0.6cm}\textbf{7. TKT Awal: 4}\hspace{1cm} \textbf{TKT akhir yang akan dihasilkan: 6}\\

\hspace{-0.6cm}\textbf{8. Grup Riset:}\\
Intelligent Systems and Robotics (ISaR)\\

\hspace{-0.6cm}\textbf{9. Instansi lain yang terlibat:}\\
Mitra\\

\hspace{-0.6cm}\textbf{10. Temuan yang ditargetkan:}
Prototipe yang dapat digunakan untuk mendemonstrasikan sistem kontrol MPC secara realtime \\

%\vspace{0.5cm}

\hspace{-0.6cm}\textbf{11. Kontribusi mendasar pada suatu bidang ilmu
(uraikan tidak lebih dari 50 kata, tekankan pada gagasan fundamental dan orisinal yang akan mendukung pengembangan IPTEKSB):}\\
Penelitian akan menghasilkan inovasi sebuah prototipe produk peralatan LAB sistem kontrol yang dapat digunakan untuk melakukan eksperimen-eksperimen disain algoritma sistem control secara real-time, dalam penelitian ini difokuskan pada implementasi Model Predictive Control.\\

\hspace{-0.6cm}\textbf{12. Jurnal ilmiah yang menjadi sasaran (tuliskan nama terbitan berkala ilmiah internasional bereputasi, nasional terakreditasi, atau nasional tidak terakreditasi dan tahun rencana publikasi): }\\
Luaran publikasi berupa artikel untuk seminar internasional terindeks Scopus, ICVEE 2024.\\

\hspace{-0.6cm}\textbf{13. Rencana luaran HKI, buku, purwarupa atau luaran lainnya yang ditargetkan, tahun rencana perolehan atau penyelesaiannya:}\\
Luaran KI berupa kode program komputer implementasi MPC untuk cart-inverted pendulum secara realtime 
\newpage
